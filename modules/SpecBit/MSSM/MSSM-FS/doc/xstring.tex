%  __________________________________________________
% |                                                  |
% |                                                  |
% |                   xstring v1.7b                  |
% |                                                  |
% |                  29 juillet 2013                 |
% |                                                  |
% |__________________________________________________|
%
% This is xtring.tex
%
% "xstring" package consists of the 7 following files:
%   xstring.tex (this file)
%   xstring.sty
%   README
%   xstring_doc_fr.tex, xstring_doc_fr.pdf (manual in french)
%   xstring_doc_en.tex, xstring_doc_en.pdf (manual in english)
%
% Christian Tellechea 2008-2013
% email : unbonpetit@gmail.com
% -------------------------------------------------------------------
% This work may be distributed and/or modified under the
% conditions of the LaTeX Project Public License, either version 1.3
% of this license or (at your option) any later version.
% The latest version of this license is in
%
%     http://www.latex-project.org/lppl.txt
%
% and version 1.3 or later is part of all distributions of LaTeX
% version 2005/12/01 or later.
% -------------------------------------------------------------------
% This work has the LPPL maintenance status `maintained'.
%
% The Current Maintainer of this work is Christian Tellechea
% -------------------------------------------------------------------
\def\xstringversion     {1.7b}
\def\xstringdate        {2013/07/29}
\def\xstringfrenchdate  {29 juillet 2013}
\def\xstringenglishdate {29 july 2013}
\edef\CurrentAtCatcode  {\the\catcode`\@}
\catcode`\@=11
\newwrite\@xs@message% canal pour les messages
\newcount\integerpart
\newcount\decimalpart% compteurs utilis\'es par xstring

\expandafter\ifx\csname @latexerr\endcsname\relax% on n'utilise pas LaTeX ?
	\immediate\write\m@ne{Package: xstring \xstringdate\space\space v\xstringversion\space\space String manipulations}%
	\long\def\@firstoftwo#1#2{#1}
	\long\def\@secondoftwo#1#2{#2}
	\long\def\@gobble#1{}
	\long\def\@ifnextchar#1#2#3{%
		\let\reserved@d=#1%
		\def\reserved@a{#2}%
		\def\reserved@b{#3}%
		\futurelet\@let@arg\@ifnch}
	\def\@ifnch{%
		\ifx\@let@arg\@sptoken
			\let\reserved@c\@xifnch
		\else
			\ifx\@let@arg\reserved@d
				\let\reserved@c\reserved@a
			\else
				\let\reserved@c\reserved@b
			\fi
		\fi
		\reserved@c}
	\def\:{\let\@sptoken= } \:
	\def\:{\@xifnch} \expandafter\def\: {\futurelet\@let@arg\@ifnch}
	\def\@ifstar#1{\@ifnextchar *{\@firstoftwo{#1}}}
	\long\def\@testopt#1#2{\@ifnextchar[{#1}{#1[{#2}]}}
	\def\@empty{}
\fi% fin des d\'efinitions LaTeX

\long\def\@xs@ifempty#1{%
	\expandafter\ifx\expandafter\relax\detokenize{#1}\relax
		\expandafter\@firstoftwo
	\else
		\expandafter\@secondoftwo
	\fi
}
% Ouvre un groupe o\`u les catcodes sont \`a 12 et \`a 10 pour les espaces
% ensuite, appelle \@xs@ReadVerb qui lit un argument entre d\'elimiteurs verb
\def\@xs@MakeVerb{% lit 1 argument et le transforme en verb
	\begingroup% groupe o\`u les catcodes sont \`a 12 pour la lecture suivante
		\def\do##1{\catcode`##112\relax}%
		\dospecials% on entre dans le mode verb
		\obeyspaces% et on tient compte des espaces
		\@xs@ReadVerb% et on va lire l'argument
}

% D\'efinit \@xs@ReadVerb qui lit un argument entre d\'elimiteurs verb
\def\setverbdelim#1{% d\'efinit quel est le d\'elimiteur de verb
	\expandafter\@xs@ifempty\expandafter{\@gobble#1}%
		\relax
		{\begingroup
			\newlinechar`\^^J%
			\immediate\write\@xs@message
			{Package xstring Warning: verb delimiter is not a single token on input line \the\inputlineno^^J}%
		\endgroup
		}%
	\def\@xs@ReadVerb##1#1##2#1{% lit ##2 qui est entre les d\'elimiteurs de verb
		\endgroup% on ferme le groupe
		\@xs@afterreadverb{##2}}% on appelle l'ex\'ecution de fin
}

% Assigne l'argument entre d\'elimiteur verb dans la sc #1' +
\def\verbtocs#1{%
	\def\@xs@afterreadverb##1{\def#1{##1}}%
	\@xs@MakeVerb
}

\begingroup
	\catcode\z@3 \def\@xs@twochars{^^00}%
	\catcode\z@7 \xdef\@xs@twochars{\@xs@twochars^^00}%
\endgroup

\edef\@xs@reserved@A{\long\def\noexpand\@xs@AssignResult##1\@xs@twochars}

\@xs@reserved@A#2{\endgroup\expandafter\def\expandafter#2\expandafter{\@gobble#1}}

\def\tokenize#1#2{%
	\begingroup
		\@xs@def\@xs@reserved@A{#2}% on d\'eveloppe en accord avec \fullexpandarg ou \noexpandarg
		\everyeof\expandafter{\@xs@twochars#1}% met "^^@^^@#1" \`a la fin du fichier virtuel
		\endlinechar\m@ne
		\expandafter\@xs@AssignResult\scantokens\expandafter{\expandafter\relax\@xs@reserved@A}% on fait l'assignation
}%

% Macro tr\`es simple qui assigne ou affiche le r\'esultat, selon la pr\'esence
% ou non de #2 qui est l'argument optionnel venant en derni\`ere position des macros
\long\def\@xs@ReturnResult#1#2{%
	\def\@xs@argument@A{#1}%
	\@xs@ifempty{#2}%
		\@xs@argument@A
		{\let#2\@xs@argument@A}%
}

\def\@xs@ldef{\long\def}

\def\@xs@ledef{\long\edef}

% Pas d'expansion des arguments
\def\normalexpandarg{%
	\let\@xs@def\@xs@ldef% on d\'efinit \@xs@call avec \def
	\def\@xs@expand##1{\unexpanded\expandafter{##1}}}
\let\noexpandarg\normalexpandarg% synonyme

% 1-d\'eveloppement du premier token des arguments
\def\expandarg{%
	\let\@xs@def\@xs@ldef% on d\'efinit \@xs@call avec \def
	\def\@xs@expand##1{\unexpanded\expandafter\expandafter\expandafter{##1}}%
}

% D\'eveloppement maximum des arguments
\def\fullexpandarg{%
	\let\@xs@def\@xs@ledef% on d\'efinit\@xs@call avec \edef
	\def\@xs@expand##1{##1}% et on neutralise \@xs@expand
}

\def\saveexpandmode{\let\@xs@saved@def\@xs@defarg\let\@xs@saved@expand\@xs@expand}

\def\restoreexpandmode{\let\@xs@defarg\@xs@saved@def\let\@xs@expand\@xs@saved@expand}

% Macro interne renvoyant #2 si son argument commence par une accolade ouvrante "{"
% et #3 sinon (Ulrich Diez sur comp.text.tex)
\long\def\@xs@ifbeginwithbrace#1{%
	\csname @%
		\expandafter\@gobble\string{%
		\expandafter\@gobble\expandafter{\expandafter{\string#1}%
		\expandafter\expandafter\expandafter\expandafter\expandafter\expandafter\expandafter\expandafter\expandafter\expandafter\expandafter\expandafter\expandafter\expandafter\expandafter\@firstoftwo
		\expandafter\expandafter\expandafter\expandafter\expandafter\expandafter\expandafter\@gobble
		\expandafter\expandafter\expandafter\@gobble
		\expandafter\expandafter\expandafter{\expandafter\string\expandafter}\string}%
		\expandafter\@gobble\string}%
		\@secondoftwo{first}{second}oftwo%
	\endcsname
}

% Cette macro interne prend la 1ere unit\'e syntaxique de #1 et assigne le r\'esultat \`a #2
\long\def\@xs@returnfirstsyntaxunit#1#2{%
	\def\@xs@groupfound{\expandafter\def\expandafter#2\expandafter{\expandafter{#2}}\@xs@gobbleall}% on met #2 dans des accolades et on finit
	\long\def\@xs@assignfirsttok##1##2\@xs@nil{\let\@xs@toks0\def#2{##1}}%
	\def\@xs@testfirsttok{%
		\let\@xs@next\@xs@assignfirsttok
		\ifx\@xs@toks\bgroup
			\expandafter\@xs@ifbeginwithbrace\expandafter{\@xs@argument}{\def\@xs@next{\afterassignment\@xs@groupfound\def#2}}{}%
		\fi
		\@xs@next}%
	\def\@xs@argument{#1}%
	\edef\@xs@next{\expandafter\@xs@beforespace\detokenize{#1} \@xs@nil}% #1 commence par un espace ?
	\ifx\@xs@next\@empty
		\def\@xs@next{\expandafter\ifx\expandafter\@empty\detokenize\expandafter{\@xs@argument}\@empty\let#2\@empty\else\def#2{ }\let\@xs@toks0\fi}%
	\else
		\def\@xs@next{\expandafter\futurelet\expandafter\@xs@toks\expandafter\@xs@testfirsttok\@xs@argument\@xs@nil}%
	\fi
	\@xs@next
}

% Cette macro interne enl\`eve la 1ere unit\'e syntaxique de #1 et assigne le r\'esultat \`a #2
\long\def\@xs@testsecondtoken#1\@xs@nil{\@xs@ifbeginwithbrace{#1}}

\long\def\@xs@gobblespacebeforebrace#1#{}% supprime tout ce qui est avant la 1ere accolade ouvrante

\long\def\@xs@removefirstsyntaxunit#1#2{%
	\def\@xs@argument{#1}%
	\expandafter\expandafter\expandafter\ifx\expandafter\expandafter\expandafter\@empty\expandafter\@xs@beforespace\detokenize\expandafter{\@xs@argument} \@xs@nil\@empty% #1 commence par un espace ?
		\expandafter\@xs@ifempty\expandafter{\@xs@argument}%
			{\let#2\@empty}
			{\afterassignment\@xs@testsecondtoken% après avoir mangé le 1er token, on va tester si la suite commence par «{»
			\expandafter\let\expandafter\@xs@secontoken\expandafter=\expandafter\@sptoken\@xs@argument\@xs@@nil\@xs@nil% on mange le 1er token et on rajoute \@xs@@nil à la fin pour éviter de perdre les accolades du groupe
				{\expandafter\expandafter\expandafter\def\expandafter\expandafter\expandafter#2%
				 \expandafter\expandafter\expandafter{\expandafter\@xs@gobblespacebeforebrace\@xs@argument}}%
				{\expandafter\expandafter\expandafter\def\expandafter\expandafter\expandafter#2%
				 \expandafter\expandafter\expandafter{\expandafter\@xs@behindspace\@xs@argument\@xs@nil}}%
			}%
	\else
		\expandafter\expandafter\expandafter\def\expandafter\expandafter\expandafter#2%
		\expandafter\expandafter\expandafter{\expandafter\@gobble\@xs@argument}%
	\fi
}

% Macros \`a arguments d\'elimit\'es pour les macros ci dessus
\long\def\@xs@beforespace#1 #2\@xs@nil{#1}

\long\def\@xs@behindspace#1 #2\@xs@nil{#2}

\long\def\@xs@returnfirstsyntaxunit@ii#1#2\@xs@nil{#1}

\long\def\@xs@gobbleall#1\@xs@nil{}

% Cette macro interne est utilis\'ee dans les macros \'etoil\'ees pour :
% 1) d\'evelopper l'argument selon qu'on a choisit \fullexpandarg
%    ou \normalexpandarg, et ceci \`a l'aide de la macro \@xs@def
% 2) Ensuite, on d\'etokenize ce d\'eveloppement de façon n'avoir plus que
%    des catcodes de 10 pour les espaces et 12 pour le reste.
\long\def\@xs@expand@and@detokenize#1#2{%
	\long\def#1{#2}%
	\expandafter\edef\expandafter#1\expandafter{\@xs@expand#1}% on d\'eveloppe #2 selon le mode de d\'eveloppement
	\long\edef#1{\detokenize\expandafter{#1}}% puis on d\'etokenize et on assigne \`a #1
}

\long\def\@xs@expand@and@assign#1#2{\@xs@def#1{#2}}% on d\'eveloppe #2 selon \fullexpandarg ou \normalexpandarg

\long\def\@xs@edefaddtomacro#1#2{\edef#1{\unexpanded\expandafter{#1}#2}}

\long\def\@xs@addtomacro#1#2{\expandafter\def\expandafter#1\expandafter{#1#2}}

\def\@xs@argstring{0########1########2########3########4########5########6########7########8########9}

% \@xs@DefArg{3} met dans la sc \@xs@myarg les tokens "####1####2####3"
\def\@xs@DefArg#1{\def\@xs@defarg0##1#1##2\@xs@nil{\def\@xs@myarg{##1#1}}\expandafter\@xs@defarg\@xs@argstring\@xs@nil}

% \@xs@DefArg@{3} met dans la sc \@xs@myarg les tokens "[####1]####2####3####4"
\def\@xs@DefArg@#1{\expandafter\@xs@defarg@\expandafter{\number\numexpr#1+1}}

\def\@xs@defarg@#1{\def\@xs@defarg0##11##2#1##3\@xs@nil{\def\@xs@myarg{[##11]##2#1}}\expandafter\@xs@defarg\@xs@argstring\@xs@nil}

% \@xs@OneArg{3} met dans la sc \@xs@myarg les tokens "####3"
\def\@xs@OneArg#1{\expandafter\@xs@onearg\expandafter{\number\numexpr#1-1}{#1}}

\def\@xs@onearg#1#2{\def\@xs@defarg##1#1##2#2##3\@xs@nil{\def\@xs@myarg{##2#2}}\expandafter\@xs@defarg\@xs@argstring\@xs@nil}

% #1 : num\'ero du 1er argument; #2 : nombre de lignes; #3 argument optionnel; #4 : dernier num\'ero de l'argument devant être d\'etokenis\'e si \'etoile
\def\@xs@BuildLines#1#2#3#4{%
	\let\@xs@newlines\@empty
	\let\@xs@newargs\@empty
	\def\@xs@buildlines##1{%
		\expandafter\@xs@OneArg\expandafter{\number\numexpr##1+#1-1}%
		\edef\@xs@reserved@B{\noexpand\@xs@expand\csname @xs@arg@\romannumeral\numexpr##1\endcsname}%
		\ifnum##1=\@ne% si c'est le premier argument
			\@xs@ifempty{#3}%
				{\expandafter\@xs@addtomacro\expandafter\@xs@newargs\expandafter{\expandafter{\@xs@reserved@B}}%
				\edef\@xs@reserved@B{\ifnum##1>#4 @xs@def\else @xs@assign\fi}%
				}% et s'il y a un argument optionnel alors, on met des crochets
				{\expandafter\@xs@addtomacro\expandafter\@xs@newargs\expandafter{\expandafter[\@xs@reserved@B]}%
				\def\@xs@reserved@B{@xs@def}% ne pas d\'etok\'eniser l'argument optionnel grace au \@xs@def
				}
		\else
			\expandafter\@xs@addtomacro\expandafter\@xs@newargs\expandafter{\expandafter{\@xs@reserved@B}}%
			\edef\@xs@reserved@B{\ifnum##1>#4 @xs@def\else @xs@assign\fi}%
		\fi
		\edef\@xs@newlines{\unexpanded\expandafter{\@xs@newlines}\expandafter\noexpand\csname\@xs@reserved@B\endcsname\expandafter\noexpand\csname @xs@arg@\romannumeral\numexpr##1\endcsname{\@xs@myarg}}%
		\ifnum##1<#2\relax
			\def\@xs@next{\expandafter\@xs@buildlines\expandafter{\number\numexpr##1+1}}%
			\expandafter\@xs@next
		\fi}%
	\@xs@buildlines\@ne
}

\def\@xs@newmacro{%
	\@ifstar
		{\let\@xs@reserved@D\@empty\@xs@newmacro@}
		{\let\@xs@reserved@D\relax\@xs@newmacro@0}%
}

% #1 : dernier num\'ero de l'argument devant être d\'etokenis\'e
% #2 : nom de la macro publique
% #3 : argument optionnel (vide si pas d'arg optionnel)
% #4 : nombre d'arguments obligatoires
% #5 : utilisation de testopt (1 si oui, 0 si non)
% #6 : code de la macro
\def\@xs@newmacro@#1#2#3#4#5{%
	\edef\@xs@reserved@A{@xs@\expandafter\@gobble\string#2}%
	\edef\@xs@reserved@C{\expandafter\noexpand\csname\@xs@reserved@A @\ifx\@empty#3\@empty @\fi\endcsname}%
	\edef\@xs@reserved@B{%
		\ifx\@empty\@xs@reserved@D
			\def\noexpand#2{\noexpand\@ifstar
				{\let\noexpand\@xs@assign\noexpand\@xs@expand@and@detokenize\expandafter\noexpand\@xs@reserved@C}%
				{\let\noexpand\@xs@assign\noexpand\@xs@expand@and@assign\expandafter\noexpand\@xs@reserved@C}%
			}%
		\else
			\def\noexpand#2{\let\noexpand\@xs@assign\noexpand\@xs@expand@and@assign\expandafter\noexpand\@xs@reserved@C}%
		\fi
		\ifx\@empty#3\@empty
		\else
			\def\expandafter\noexpand\@xs@reserved@C{%
				\noexpand\@testopt{\expandafter\noexpand\csname\@xs@reserved@A @@\endcsname}{\ifx\@xs@def\@xs@ledef#3\else\unexpanded{#3}\fi}}%
		\fi
	}%
	% Que fait \@xs@reserved@B ? Il d\'efinit :
	% si #3 est vide : \NOM{\@ifstar{\let\@xs@assign\@xs@expand@and@detokenize\@xs@NOM@@}{\let\@xs@assign\@xs@expand@and@assign\@xs@NOM@@}}
	% si #3 existe   : \NOM{\@ifstar{\let\@xs@assign\@xs@expand@and@detokenize\@xs@NOM@}{\let\@xs@assign\@xs@expand@and@assign\@xs@NOM@}}
	%                  \@xs@NOM@{\@testopt{\@xs@NOM@@}{#3}}}
	\long\@xs@reserved@B
	\ifx\@empty#3\@empty
		\@xs@BuildLines1{#4}{#3}{#1}%
		\@xs@DefArg{#4}%
	\else
		\expandafter\@xs@BuildLines\expandafter1\expandafter{\number\numexpr#4+1}{#3}{#1}%
		\@xs@DefArg@{#4}%
	\fi
	\edef\@xs@reserved@B{\def\expandafter\noexpand\csname\@xs@reserved@A @@\endcsname\@xs@myarg}%
	\edef\@xs@reserved@C{\unexpanded\expandafter{\@xs@newlines}\edef\noexpand\@xs@call}%
	\edef\@xs@reserved@D{%
		\noexpand\noexpand\expandafter\noexpand\csname\@xs@reserved@A\endcsname\unexpanded\expandafter{\@xs@newargs}%
	}%
	\ifnum#5=\@ne\edef\@xs@reserved@D{\noexpand\noexpand\noexpand\@testopt{\unexpanded\expandafter{\@xs@reserved@D}}{}}\fi
	\@xs@edefaddtomacro\@xs@reserved@C{{\unexpanded\expandafter{\@xs@reserved@D}}\noexpand\@xs@call}%
	\@xs@edefaddtomacro\@xs@reserved@B{{\unexpanded\expandafter{\@xs@reserved@C}}}%
	% Que fait \@xs@reserved@B ? Il d\'efinit par exemple pour 3 arguments obligatoires et 1 facultatif :
	%    \def\@xs@NOM@@[##2]##3##4##5{%
	%          \@xs@def\@xs@arg@i{##2}\@xs@assign\@xs@arg@ii{##3}\@xs@assign\@xs@arg@iii{##4}\@xs@asign\@xs@arg@iv{##5}%
	% si #5=0: \edef\@xs@call{\noexpand\@xs@NOM[\@xs@expand\@xs@arg@i]{\@xs@expand\@xs@arg@ii}{\@xs@expand\@xs@arg@iii}{\@xs@expand\@xs@arg@iv}}%
	% si #5=1: \edef\@xs@call{\noexpand\@testopt{\noexpand\@xs@NOM[\@xs@expand\@xs@arg@i]{\@xs@expand\@xs@arg@i}{\@xs@expand\@xs@arg@ii}{\@xs@expand\@xs@arg@iii}{\@xs@expand\@xs@arg@iv}}{}}%
	%          \@xs@call}
	\long\@xs@reserved@B
	\edef\@xs@reserved@B{%
		\def\expandafter\noexpand\csname\@xs@reserved@A\endcsname
			\@xs@myarg\ifnum#5=\@ne[\unexpanded{##}\number\numexpr\ifx\@empty#3\@empty#4+1\else#4+2\fi]\fi
	}%
	% Que fait \@xs@reserved@B ? Il d\'efinit par exemple pour 3 arguments obligatoires et 1 facultatif :
	%    \def \@xs@NOM[##2]##3##4##5[##6]{#6}
	\long\@xs@reserved@B
}

% macro g\'en\'erique qui lit \@xs@reserved@C us par us
% 3 sous-routines sont appel\'ees \`a des moments cl\'e :
% \@xs@atendofgroup      (un groupe se finit, appel r\'ecursif)
% \@xs@atbegingroup      (un groupe vient d'être ouvert)
% \@xs@atnextsyntaxunit  (la future US n'est pas un groupe)
\def\@xs@read@reserved@C{%
	\expandafter\@xs@ifempty\expandafter{\@xs@reserved@C}%
		{\ifnum\@xs@nestlevel=\z@
			\let\@xs@next\relax
		\else
			\let\@xs@next\@xs@atendofgroup
		\fi
		}
		{\expandafter\@xs@returnfirstsyntaxunit\expandafter{\@xs@reserved@C}\@xs@reserved@A
		\expandafter\@xs@removefirstsyntaxunit\expandafter{\@xs@reserved@C}\@xs@reserved@C
		\let\@xs@next\@xs@read@reserved@C
		\@xs@exploregroups
		\ifx\bgroup\@xs@toks
			\advance\integerpart\@ne
			\begingroup
				\expandafter\def\expandafter\@xs@reserved@C\@xs@reserved@A
				\@xs@manage@groupID
				\let\@xs@nestlevel\@ne
				\integerpart\z@
				\@xs@atbegingroup
		\else
			\global\advance\decimalpart\@ne
			\@xs@atnextsyntaxunit
		\fi
		}%
	\@xs@next
}

% macro g\'en\'erique qui lit \@xs@reserved@D en proc\'edant \`a des tests avec \IfBeginWith
% 2 sous-routines sont appel\'ees \`a des moments cl\'e :
% \@xs@atendofgroup      (un groupe se finit, appel r\'ecursif)
% \@xs@atoccurfound      (une occurrence a \'et\'e trouv\'ee)
\def\@xs@read@reserved@D{%
	\expandafter\@xs@ifempty\expandafter{\@xs@reserved@D}%
		{\ifnum\@xs@nestlevel=\z@
			\let\@xs@next\relax
		\else
			\let\@xs@next\@xs@atendofgroup
		\fi
		}%
		{\expandafter\expandafter\expandafter\@xs@IfBeginWith@i\expandafter\expandafter\expandafter
			{\expandafter\@xs@reserved@D\expandafter}\expandafter{\@xs@reserved@E}%
			{\global\advance\decimalpart\@ne
			\let\@xs@reserved@D\@xs@reserved@A
			\@xs@atoccurfound
			}%
			{\expandafter\@xs@returnfirstsyntaxunit\expandafter{\@xs@reserved@D}\@xs@reserved@A
			\expandafter\@xs@removefirstsyntaxunit\expandafter{\@xs@reserved@D}\@xs@reserved@D
			\let\@xs@next\@xs@read@reserved@D
			\@xs@exploregroups
			\ifx\bgroup\@xs@toks
				\advance\integerpart\@ne
				\begingroup
					\expandafter\def\expandafter\@xs@reserved@D\@xs@reserved@A
					\@xs@manage@groupID
					\let\@xs@reserved@C\@empty
					\let\@xs@nestlevel\@ne
					\integerpart\z@
			\else
				\expandafter\@xs@addtomacro\expandafter\@xs@reserved@C\expandafter{\@xs@reserved@A}%
			\fi
			}%
		}%
	\@xs@next
}

\@xs@newmacro\StrRemoveBraces{}{1}{1}{%
	\def\@xs@reserved@C{#1}%
	\let\@xs@reserved@B\@empty
	\let\@xs@nestlevel\z@
	\@xs@StrRemoveBraces@i
	\expandafter\@xs@ReturnResult\expandafter{\@xs@reserved@B}{#2}%
}

\def\@xs@StrRemoveBraces@i{%
	\expandafter\@xs@ifempty\expandafter{\@xs@reserved@C}%
		{\ifnum\@xs@nestlevel=\z@
			\let\@xs@next\relax
		\else
			\expandafter\endgroup
			\expandafter\@xs@addtomacro\expandafter\@xs@reserved@B\expandafter{\@xs@reserved@B}%
			\let\@xs@next\@xs@StrRemoveBraces@i
		\fi
		}
		{\expandafter\@xs@returnfirstsyntaxunit\expandafter{\@xs@reserved@C}\@xs@reserved@A
		\expandafter\@xs@removefirstsyntaxunit\expandafter{\@xs@reserved@C}\@xs@reserved@C
		\let\@xs@next\@xs@StrRemoveBraces@i
		\ifx\bgroup\@xs@toks
			\ifx\@xs@exploregroups\relax% on explore les groupes ?
				\begingroup
					\expandafter\def\expandafter\@xs@reserved@C\@xs@reserved@A
					\let\@xs@nestlevel\@ne
					\integerpart\z@
					\let\@xs@reserved@B\@empty
			\else
				\expandafter\@xs@addtomacro\expandafter\@xs@reserved@B\@xs@reserved@A
			\fi
		\else
			\expandafter\@xs@addtomacro\expandafter\@xs@reserved@B\expandafter{\@xs@reserved@A}%
		\fi
		}%
	\@xs@next
}

% \@xs@cutafteroccur coupe l'argument #1 apr\`es la #3\`eme occurrence de #2
% \@xs@reserved@C : contient ce qui se trouve avant cette occurrence
% \@xs@reserved@D : contient ce qui se trouve avant cette occurrence y compris cette occurrence
% \@xs@reserved@E : contient ce qui se trouve apr\`es l'occurrence
% si l'occurrence n'existe pas ou qu'un des arguments est vide, toutes les chaines renvoy\'ees sont vides
\long\def\@xs@cutafteroccur#1#2#3{%
	\ifnum#3<\@ne\expandafter\@firstoftwo\else\expandafter\@secondoftwo\fi
		{\let\@xs@reserved@C\@empty
		\let\@xs@reserved@E\@empty
		\global\let\groupID\@empty
		}
		{\@xs@cutafteroccur@i{#1}{#2}{#3}}%
}
	
\long\def\@xs@cutafteroccur@i#1#2#3{%
	\def\@xs@reserved@D{#1}\let\@xs@reserved@C\@empty\def\@xs@reserved@E{#2}%
	\decimalpart\z@
	\integerpart\z@
	\gdef\groupID{0}%
	\let\@xs@nestlevel\z@
	\def\@xs@atendofgroup{%
		\expandafter\endgroup
		\expandafter\@xs@addtomacro\expandafter\@xs@reserved@C\expandafter{\expandafter{\@xs@reserved@C}}%
		\@xs@read@reserved@D}%
	\def\@xs@atoccurfound{%
		\ifnum\decimalpart=\numexpr(#3)\relax
			\global\let\@xs@reserved@D\@xs@reserved@D
			\global\let\@xs@reserved@C\@xs@reserved@C
			\@xs@exitallgroups
			\let\@xs@next\relax
		\else
			\expandafter\@xs@addtomacro\expandafter\@xs@reserved@C\expandafter{\@xs@reserved@E}%
			\let\@xs@next\@xs@read@reserved@D
		\fi}%
	\@xs@read@reserved@D
	\def\@xs@argument@A{#2}%
	\ifnum\decimalpart=\numexpr(#3)\relax % occurrence trouv\'ee ?
		\let\@xs@reserved@E\@xs@reserved@D
		\expandafter\expandafter\expandafter\def\expandafter\expandafter\expandafter\@xs@reserved@D\expandafter\expandafter\expandafter{\expandafter\@xs@reserved@C\@xs@argument@A}%
	\else
		\let\@xs@reserved@C\@empty
		\let\@xs@reserved@E\@empty
		\global\let\groupID\@empty
	\fi
}

\@xs@newmacro*3\IfSubStr{1}{2}{0}{%
	\def\@xs@argument@A{#2}\def\@xs@argument@B{#3}%
	\expandafter\expandafter\expandafter\@xs@cutafteroccur
	\expandafter\expandafter\expandafter{\expandafter\@xs@argument@A\expandafter}\expandafter{\@xs@argument@B}{#1}%
	\expandafter\@xs@ifempty\expandafter{\@xs@reserved@D}%
		\@secondoftwo
		\@firstoftwo
}

\@xs@newmacro*2\IfBeginWith{}{2}{0}{%
	\def\@xs@argument@A{#1}\def\@xs@argument@B{#2}%
	\expandafter\@xs@ifempty\expandafter{\@xs@argument@B}%
		{\let\@xs@next\@secondoftwo
		}
		{\def\@xs@next{%
			\expandafter\expandafter\expandafter\@xs@IfBeginWith@i\expandafter\expandafter\expandafter
				{\expandafter\@xs@argument@A\expandafter}\expandafter
				{\@xs@argument@B}}%
		}%
	\@xs@next
}

\long\def\@xs@IfBeginWith@i#1#2{%
	\def\@xs@argument@A{#1}\def\@xs@argument@B{#2}%
	\expandafter\@xs@ifempty\expandafter{\@xs@argument@B}%
		{\let\@xs@next\@firstoftwo% #2 est vide, tous les tests sont pass\'es avec succ\`es : on renvoie #3
		}
		{\expandafter\@xs@ifempty\expandafter{\@xs@argument@A}%
			{\let\@xs@next\@secondoftwo% #1 est vide, c'est que #2 est + long que #1 : on renvoie #4
			}
			{\expandafter\@xs@returnfirstsyntaxunit\expandafter{\@xs@argument@B}\@xs@reserved@B
			\expandafter\@xs@returnfirstsyntaxunit\expandafter{\@xs@argument@A}\@xs@reserved@A
			\ifx\@xs@reserved@A\@xs@reserved@B% il y a \'egalit\'e...
				\expandafter\@xs@removefirstsyntaxunit\expandafter{\@xs@argument@B}\@xs@reserved@B
				\expandafter\@xs@removefirstsyntaxunit\expandafter{\@xs@argument@A}\@xs@reserved@A% on enl\`eve les 1ere unit\'es syntaxiques
				\def\@xs@next{% et on recommence avec ces arguments racourcis d'1 unit\'e syntaxique
					\expandafter\expandafter\expandafter\@xs@IfBeginWith@i
					\expandafter\expandafter\expandafter{\expandafter\@xs@reserved@A\expandafter}\expandafter{\@xs@reserved@B}}%
			\else
				\let\@xs@next\@secondoftwo
			\fi
			}%
		}%
	\@xs@next
}

\@xs@newmacro*2\IfEndWith{}{2}{0}{%
	\def\@xs@argument@A{#1}\def\@xs@argument@B{#2}%
	\@xs@ifempty{#2}%
		{\let\@xs@reserved@A\@secondoftwo
		}
		{\expandafter\expandafter\expandafter\@xs@StrCount
			\expandafter\expandafter\expandafter{\expandafter\@xs@argument@A\expandafter}\expandafter
			{\@xs@argument@B}[\@xs@reserved@A]%
		\ifnum\@xs@reserved@A=\z@
			\let\@xs@reserved@A\@secondoftwo
		\else
			\expandafter\@xs@ifempty\expandafter{\@xs@reserved@C}%
				{\let\@xs@reserved@A\@firstoftwo}
				{\let\@xs@reserved@A\@secondoftwo}%
		\fi
		}%
	\@xs@reserved@A
}

\@xs@newmacro*4\IfSubStrBefore{1,1}{3}{0}{\@xs@IfSubStrBefore@i[#1]{#2}{#3}{#4}}

\def\@xs@IfSubStrBefore@i[#1,#2]#3#4#5{%
	\def\@xs@reserved@C{#3}%
	\ifx\@xs@exploregroups\relax% si on explore les groupes
		\let\@xs@reserved@B\@empty
		\let\@xs@nestlevel\z@
		\@xs@StrRemoveBraces@i% on retire les accolades
		\let\@xs@reserved@C\@xs@reserved@B
	\fi
	\def\@xs@reserved@A{#5}%
	\expandafter\expandafter\expandafter\@xs@cutafteroccur\expandafter\expandafter\expandafter{\expandafter\@xs@reserved@C\expandafter}\expandafter{\@xs@reserved@A}{#2}%
	\def\@xs@reserved@A{#4}%
	\expandafter\expandafter\expandafter\@xs@cutafteroccur\expandafter\expandafter\expandafter{\expandafter\@xs@reserved@C\expandafter}\expandafter{\@xs@reserved@A}{#1}%
	\global\let\groupID\@empty
	\expandafter\@xs@ifempty\expandafter{\@xs@reserved@C}%
		\@secondoftwo
		\@firstoftwo
}

\@xs@newmacro*4\IfSubStrBehind{1,1}{3}{0}{\@xs@IfSubStrBehind@i[#1]{#2}{#3}{#4}}

\long\def\@xs@IfSubStrBehind@i[#1,#2]#3#4#5{\@xs@IfSubStrBefore@i[#2,#1]{#3}{#5}{#4}}

\long\def\@xs@formatnumber#1#2{%
	\def\@xs@argument@A{#1}%
	\@xs@ifempty{#1}%
		{\def#2{0X}% si vide, renvoie 0X
		}
		{\@xs@returnfirstsyntaxunit{#1}\@xs@reserved@A
		\def\@xs@reserved@B{+}%
		\ifx\@xs@reserved@A\@xs@reserved@B
			\expandafter\@xs@removefirstsyntaxunit\expandafter{\@xs@argument@A}\@xs@reserved@C
			\expandafter\@xs@ifempty\expandafter{\@xs@reserved@C}%
			 	{\def#2{+0X}}
			 	{\expandafter\def\expandafter#2\expandafter{\expandafter+\expandafter0\@xs@reserved@C}}%
		\else
			\def\@xs@reserved@B{-}%
			\ifx\@xs@reserved@A\@xs@reserved@B
				\expandafter\@xs@removefirstsyntaxunit\expandafter{\@xs@argument@A}\@xs@reserved@A
				\expandafter\@xs@ifempty\expandafter{\@xs@reserved@A}%
					{\def#2{-0X}}%
					{\expandafter\def\expandafter#2\expandafter{\expandafter-\expandafter0\@xs@reserved@A}}%
			\else
				\expandafter\def\expandafter#2\expandafter{\expandafter0\@xs@argument@A}%
			\fi
		\fi
		}%
}

\@xs@newmacro\IfInteger{}{1}{0}{%
	\@xs@formatnumber{#1}\@xs@reserved@A
	\decimalpart\z@
	\afterassignment\@xs@defafterinteger\integerpart\@xs@reserved@A\relax\@xs@nil
	\let\@xs@after@intpart\@xs@afterinteger
	\expandafter\@xs@testdot\@xs@afterinteger\@xs@nil
	\ifx\@empty\@xs@afterdecimal
		\ifnum\decimalpart=\z@
			\let\@xs@next\@firstoftwo% partie décimale constituée de 0 --> seul cas où on renvoie vrai
		\else
			\let\@xs@afterinteger\@xs@after@intpart
			\let\@xs@next\@secondoftwo
		\fi
	\else
		\let\@xs@afterinteger\@xs@after@intpart
		\let\@xs@next\@secondoftwo
	\fi
	\@xs@next
}

\@xs@newmacro\IfDecimal{}{1}{0}{%
	\@xs@formatnumber{#1}\@xs@reserved@A
	\decimalpart\z@
	\afterassignment\@xs@defafterinteger\integerpart\@xs@reserved@A\relax\@xs@nil
	\expandafter\@xs@testdot\@xs@afterinteger\@xs@nil
	\ifx\@empty\@xs@afterdecimal
		\expandafter\@firstoftwo
	\else
		\expandafter\@secondoftwo
	\fi
}

\long\def\@xs@defafterinteger#1\relax\@xs@nil{\def\@xs@afterinteger{#1}}

\def\@xs@testdot{%
	\let\xs@decsep\@empty
	\@ifnextchar.%
		{\def\xs@decsep{.}\@xs@readdecimalpart}%
		{\@xs@testcomma}%
}

\def\@xs@testcomma{%
	\@ifnextchar,%
		{\def\xs@dessep{,}\@xs@readdecimalpart}%
		{\@xs@endnumber}%
}

\long\def\@xs@readdecimalpart#1#2\@xs@nil{%
	\ifx\@empty#2\@empty
		\def\@xs@reserved@A{0X}%
	\else
		\def\@xs@reserved@A{0#2}%
	\fi
	\afterassignment\@xs@defafterinteger\decimalpart\@xs@reserved@A\relax\@xs@nil
	\expandafter\@xs@endnumber\@xs@afterinteger\@xs@nil
}

\long\def\@xs@endnumber#1\@xs@nil{\def\@xs@afterdecimal{#1}}

% test d'\'egalit\'e sur des chaines (et des chaines contenant des nombres)
\long\def\@xs@IfStrEqFalse@i#1#2{\let\@xs@reserved@A\@secondoftwo}
\long\def\@xs@IfStrEqFalse@ii#1#2{% renvoie vrai si les 2 arg sont d\'ecimaux et s'ils sont \'egaux, faux sinon
	\@xs@IfDecimal{#1}%
		{\@xs@IfDecimal{#2}%
			{\ifdim#1pt=#2pt
				\let\@xs@reserved@A\@firstoftwo
			\else
				\let\@xs@reserved@A\@secondoftwo
			\fi% les 2 sont d\'ecimaux : on fait le test
			}%
			{\let\@xs@reserved@A\@secondoftwo}% un seul est d\'ecimal
		}%
		{\let\@xs@reserved@A\@secondoftwo}% #1 n'est pas d\'ecimal
}

\long\def\@xs@TestEqual#1#2{% teste si les 2 arguments sont \'egaux
	\def\@xs@reserved@A{#1}\def\@xs@reserved@B{#2}%
	\ifx\@xs@reserved@A\@xs@reserved@B
		\let\@xs@reserved@A\@firstoftwo% \'egalit\'e parfaite des 2 chaines
	\else
		\expandafter\expandafter\expandafter\@xs@reserved@D\expandafter\expandafter\expandafter{\expandafter\@xs@reserved@A\expandafter}\expandafter{\@xs@reserved@B}%
	\fi
	\@xs@reserved@A
}

\@xs@newmacro*2\IfStrEq{}{2}{0}{% teste si les deux chaines sont \'egales
	\let\@xs@reserved@D\@xs@IfStrEqFalse@i
	\@xs@TestEqual{#1}{#2}%
}

\@xs@newmacro*2\IfEq{}{2}{0}{% teste si les 2 arguments (chaine ou nombre) sont \'egaux
	\let\@xs@reserved@D\@xs@IfStrEqFalse@ii
	\@xs@TestEqual{#1}{#2}%
}

\def\IfStrEqCase{%
	\@ifstar
		{\def\@xs@reserved@E{\IfStrEq*}\@xs@IfStrCase}%
		{\def\@xs@reserved@E{\IfStrEq}\@xs@IfStrCase}%
}

\def\@xs@IfStrCase#1#2{\@testopt{\@xs@IfStringCase{#1}{#2}}{}}

\def\IfEqCase{%
	\@ifstar
		{\def\@xs@reserved@E{\IfEq*}\@xs@IfEqCase}%
		{\def\@xs@reserved@E{\IfEq}\@xs@IfEqCase}%
}

\long\def\@xs@IfEqCase#1#2{\@testopt{\@xs@IfStringCase{#1}{#2}}{}}

\long\def\@xs@IfStringCase#1#2[#3]{%
	\def\@xs@testcase##1##2##3\@xs@nil{% lit les 2 premieres unit\'es syntaxiques dans ##1 et ##2. Les autres dans ##3
		\@xs@reserved@E{#1}{##1}%
			{##2}% le test est positif, on ex\'ecute le code correspondant
			{\@xs@ifempty{##3}%
			 	{#3}% s'il n'y a plus de cas, on ex\'ecute le code <autres cas>
				{\@xs@testcase##3\@xs@nil}% sinon, on recommence avec ce qui reste
			 }%
	}%
	\@xs@testcase#2\@xs@nil
}

% Renvoie ce qui est \`a gauche de l'occurence n°#1 de la sous chaine #3 dans la chaine #2
\@xs@newmacro*3\StrBefore{1}{2}{1}{%
	\@xs@cutafteroccur{#2}{#3}{#1}%
	\expandafter\@xs@ReturnResult\expandafter{\@xs@reserved@C}{#4}%
}

% Renvoie ce qui est \`a droite de l'occurence n°#1 de la sous chaine #3 dans la chaine #2
\@xs@newmacro*3\StrBehind{1}{2}{1}{%
	\@xs@cutafteroccur{#2}{#3}{#1}%
	\expandafter\@xs@ReturnResult\expandafter{\@xs@reserved@E}{#4}%
}

% Renvoie ce qui est strictement compris entre les occurrences n°#1 et n°#2
% des chaines #4 et #5 dans la chaine #3
\@xs@newmacro*4\StrBetween{1,1}{3}{1}{\@xs@StrBetween@i[#1]{#2}{#3}{#4}[#5]}

\long\def\@xs@StrBetween@i[#1,#2]#3#4#5[#6]{%
	\begingroup
		\noexploregroups
		\@xs@cutafteroccur{#3}{#5}{#2}%
		\expandafter\@xs@cutafteroccur\expandafter{\@xs@reserved@C}{#4}{#1}%
		\expandafter
	\endgroup
	\expandafter\@xs@ReturnResult\expandafter{\@xs@reserved@E}{#6}%
	\global\let\groupID\@empty
}

\def\exploregroups{\let\@xs@exploregroups\relax}

\def\noexploregroups{\def\@xs@exploregroups{\let\@xs@toks0\relax}}

\def\saveexploremode{\let\@xs@saveexploremode\@xs@exploregroups}

\def\restoreexploremode{\let\@xs@exploregroups\@xs@saveexploremode}

% Remplace les #1 premi\`eres occurences de la chaine #3
% par la chaine #4 dans la chaine #2
\@xs@newmacro\StrSubstitute{0}{3}{1}{%
	\def\@xs@reserved@D{#2}\let\@xs@reserved@C\@empty\def\@xs@reserved@E{#3}%
	\def\@xs@argument@C{#3}\def\@xs@argument@D{#4}%
	\decimalpart\z@
	\integerpart\z@
	\gdef\groupID{0}%
	\let\@xs@nestlevel\z@
	\def\@xs@atendofgroup{%
		\expandafter\endgroup
		\expandafter\@xs@addtomacro\expandafter\@xs@reserved@C\expandafter{\expandafter{\@xs@reserved@C}}%
		\@xs@read@reserved@D
	}%
	\def\@xs@atoccurfound{%
		\ifnum#1<\@ne
			\let\@xs@reserved@A\@xs@argument@D
		\else
			\ifnum\decimalpart>#1
				\let\@xs@reserved@A\@xs@argument@C
			\else
				\let\@xs@reserved@A\@xs@argument@D
			\fi
		\fi
		\expandafter\@xs@addtomacro\expandafter\@xs@reserved@C\expandafter{\@xs@reserved@A}%
		\@xs@read@reserved@D
	}%
	\@xs@ifempty{#3}%
		{\expandafter\@xs@ReturnResult\expandafter{\@xs@reserved@D}{#5}%
		}
		{\@xs@read@reserved@D
		\expandafter\@xs@ReturnResult\expandafter{\@xs@reserved@C}{#5}%
		}%
}

% Supprime les #1 premi\`eres occurrences de #3 dans #2
\@xs@newmacro\StrDel{0}{2}{1}{\@xs@StrSubstitute[#1]{#2}{#3}{}[#4]}

\def\@xs@exitallgroups{\ifnum\@xs@nestlevel>\z@\endgroup\expandafter\@xs@exitallgroups\fi}

% Compte combien d'unit\'es syntaxiques contient la chaine #1 ?
\@xs@newmacro\StrLen{}{1}{1}{%
	\def\@xs@reserved@C{#1}%
	\decimalpart\z@
	\let\@xs@nestlevel\z@
	\gdef\groupID{0}%
	\let\@xs@atbegingroup\relax
	\def\@xs@atendofgroup{\endgroup\@xs@read@reserved@C}%
	\let\@xs@atnextsyntaxunit\relax
	\@xs@read@reserved@C
	\expandafter\@xs@ReturnResult\expandafter{\number\decimalpart}{#2}%
}

% Macro interne fermant autant de groupes que n\'ecessaire pour trouver une unit\'e syntaxique derri\`ere
\def\@xs@continuetonext{%
	\expandafter\@xs@ifempty\expandafter{\@xs@reserved@C}%
		{\ifnum\@xs@nestlevel>\z@
			\expandafter\endgroup\expandafter\@xs@addtomacro\expandafter\@xs@reserved@B\expandafter{\expandafter{\@xs@reserved@B}}% on concat\`ene
			\expandafter\expandafter\expandafter\@xs@continuetonext% on recommence
		\fi}
		\relax
}%

\def\@xs@manage@groupID{%
	\begingroup\def\@xs@reserved@A{0}%
	\ifx\@xs@reserved@A\groupID% si on arrive du groupe de niveau 0
		\endgroup
		\xdef\groupID{\number\integerpart}% on met simplement le niveau
	\else
		\endgroup
		\expandafter\expandafter\expandafter\gdef\expandafter\expandafter\expandafter\groupID\expandafter\expandafter\expandafter{\expandafter\groupID\expandafter,\number\integerpart}%
	\fi
}

% Cette macro coupe la chaine #1 apr\`es l'unit\'e syntaxique n°#2
\def\StrSplit{%
	\@ifstar
		{\let\@xs@reserved@E\@xs@continuetonext\StrSpl@t}%
		{\let\@xs@reserved@E\relax\StrSpl@t}%
}

\@xs@newmacro\StrSpl@t{}{2}{0}{\@xs@StrSplit@i{#2}{#1}\@xs@StrSplit@ii}

% Cette macro interne coupe la chaine #2 apr\`es l'unit\'e syntaxique n°#1
% Le d\'ebut est assign\'e dans \@xs@reserved@B et la fin dans \@xs@reserved@C
\long\def\@xs@StrSplit@i#1#2{%
	\def\@xs@reserved@D{#1}%
	\def\@xs@reserved@C{#2}%
	\let\@xs@reserved@B\@empty
	\global\let\groupID\@empty
	\ifnum#1>\z@
		\decimalpart\z@
		\integerpart\z@
		\gdef\groupID{0}%
		\let\@xs@nestlevel\z@
		\def\@xs@atendofgroup{%
			\expandafter\endgroup
			\expandafter\@xs@addtomacro\expandafter\@xs@reserved@B\expandafter{\expandafter{\@xs@reserved@B}}%
			\@xs@read@reserved@C
		}%
		\def\@xs@atbegingroup{\let\@xs@reserved@B\@empty}%
		\def\@xs@atnextsyntaxunit{%
			\expandafter\@xs@addtomacro\expandafter\@xs@reserved@B\expandafter{\@xs@reserved@A}%
			\ifnum\decimalpart=\@xs@reserved@D\relax
				\ifx\@xs@reserved@C\@empty\@xs@reserved@E\fi
				\global\let\@xs@reserved@B\@xs@reserved@B
				\global\let\@xs@reserved@C\@xs@reserved@C
				\@xs@exitallgroups
				\let\@xs@next\relax
			\fi
		}%
		\@xs@read@reserved@C
	\fi
}

\def\@xs@StrSplit@ii#1#2{\let#1\@xs@reserved@B\let#2\@xs@reserved@C}

% StrCut[n]{<texte>}{<motif>}\macroA\macroB
% \macroA reçoit ce qui se trouve avant la n ème occurrence <motif> dans <texte>
% \macroB reçoit ce qui est après cette n ème occurrence
\@xs@newmacro*3\StrCut{1}{2}{0}{%
	\@xs@ifempty{#3}%
		{\global\let\groupID\@empty
		\let\@xs@reserved@C\@empty
		\let\@xs@reserved@E\@empty
		}
		{\ifnum#1<\@ne\expandafter\@firstoftwo\else\expandafter\@secondoftwo\fi
			{\@xs@StrCut@ii{#2}{#3}1}
			{\@xs@StrCut@ii{#2}{#3}{#1}}%
		}%
	\@xs@StrCut@iii
}

\long\def\@xs@StrCut@ii#1#2#3{%
	\def\@xs@reserved@D{#1}%
	\let\@xs@reserved@C\@empty
	\def\@xs@reserved@E{#2}%
	\decimalpart\z@\integerpart\z@
	\gdef\groupID{0}%
	\let\@xs@nestlevel\z@
	\def\@xs@atendofgroup{%
		\expandafter\endgroup
		\expandafter\@xs@addtomacro\expandafter\@xs@reserved@C\expandafter{\expandafter{\@xs@reserved@C}}%
		\@xs@read@reserved@D
	}%
	\def\@xs@atoccurfound{%
		\ifnum\decimalpart=\numexpr(#3)\relax
			\global\let\@xs@reserved@D\@xs@reserved@D
			\global\let\@xs@reserved@C\@xs@reserved@C
			\@xs@exitallgroups
			\let\@xs@next\relax
		\else
				\expandafter\@xs@addtomacro\expandafter\@xs@reserved@C\expandafter{\@xs@reserved@E}%
			\let\@xs@next\@xs@read@reserved@D
		\fi
	}%
	\@xs@read@reserved@D
	\def\@xs@argument@A{#2}%
	\let\@xs@reserved@E\@xs@reserved@D
		\expandafter\expandafter\expandafter
	\def
		\expandafter\expandafter\expandafter
	\@xs@reserved@D
		\expandafter\expandafter\expandafter
	{\expandafter\@xs@reserved@C\@xs@argument@A}%
}

\long\def\@xs@StrCut@iii#1#2{\let#1\@xs@reserved@C\let#2\@xs@reserved@E}

% De la chaine #1, renvoie ce qui se trouve entre les positions
%  #2 et #3, unit\'es syntaxiques aux positions compris !
\@xs@newmacro\StrMid{}{3}{1}{%
	\begingroup
		\noexploregroups
		\let\@xs@reserved@E\relax
		\@xs@StrSplit@i{#3}{#1}%
		\edef\@xs@reserved@C{\number\numexpr#2-1}%
		\let\@xs@reserved@E\relax
		\expandafter\expandafter\expandafter\@xs@StrSplit@i\expandafter\expandafter\expandafter{\expandafter\@xs@reserved@C\expandafter}\expandafter{\@xs@reserved@B}%
	\expandafter\endgroup
	\expandafter\@xs@ReturnResult\expandafter{\@xs@reserved@C}{#4}%
	\global\let\groupID\@empty
}

% supprime #2 unit\'es syntaxiques \`a gauche dans la chaine #1
\@xs@newmacro\StrGobbleLeft{}{2}{1}{%
	\let\@xs@reserved@E\relax
	\@xs@StrSplit@i{#2}{#1}%
	\expandafter\@xs@ReturnResult\expandafter{\@xs@reserved@C}{#3}%
}

% extrait de #1 la chaine \`a gauche de longueur #2
\@xs@newmacro\StrLeft{}{2}{1}{%
	\let\@xs@reserved@E\relax
	\@xs@StrSplit@i{#2}{#1}%
	\expandafter\@xs@ReturnResult\expandafter{\@xs@reserved@B}{#3}%
}

% supprime #2 unit\'es syntaxiques \`a droite dans la chaine #1
\@xs@newmacro\StrGobbleRight{}{2}{1}{%
	\@xs@StrLen{#1}[\@xs@reserved@D]%
	\let\@xs@reserved@E\relax
	\expandafter\@xs@StrSplit@i\expandafter{\number\numexpr\@xs@reserved@D-#2}{#1}%
	\expandafter\@xs@ReturnResult\expandafter{\@xs@reserved@B}{#3}%
}

% renvoie #2 unit\'es syntaxiques \`a la droite de la chaine #1
\@xs@newmacro\StrRight{}{2}{1}{%
	\@xs@StrLen{#1}[\@xs@reserved@D]%
	\let\@xs@reserved@E\relax
	\expandafter\@xs@StrSplit@i\expandafter{\number\numexpr\@xs@reserved@D-#2}{#1}%
	\expandafter\@xs@ReturnResult\expandafter{\@xs@reserved@C}{#3}%
}

\@xs@newmacro\StrChar{}{2}{1}{%
	\let\@xs@reserved@B\@empty
	\def\@xs@reserved@C{#1}\def\@xs@reserved@D{#2}%
	\ifnum#2>\z@
		\def\groupID{0}\let\@xs@nestlevel\z@\integerpart\z@\decimalpart\z@
		\let\@xs@atbegingroup\relax
		\def\@xs@atendofgroup{\endgroup\@xs@read@reserved@C}%
		\def\@xs@atnextsyntaxunit{%
			\ifnum\decimalpart=\@xs@reserved@D% la n i\`eme US est atteinte ?
				\global\let\@xs@reserved@B\@xs@reserved@A% on capture l'US en cours qui est celle cherch\'ee
				\@xs@exitallgroups
				\let\@xs@next\relax
			\fi
		}%
		\@xs@read@reserved@C
	\fi
	\expandafter\@xs@ifempty\expandafter{\@xs@reserved@B}%
		{\global\let\groupID\@empty}
		\relax
	\expandafter\@xs@ReturnResult\expandafter{\@xs@reserved@B}{#3}%
}

% Combien de fois compte t-on #2 dans #1 ?
\@xs@newmacro\StrCount{}{2}{1}{%
	\def\@xs@reserved@D{#1}\def\@xs@reserved@E{#2}\let\@xs@reserved@C\@empty
	\@xs@ifempty{#2}%
		{\@xs@ReturnResult{0}{#3}}
		{\decimalpart\z@
		\integerpart\z@
		\gdef\groupID{0}%
		\let\@xs@nestlevel\z@
		\def\@xs@atendofgroup{%
			\expandafter\endgroup
			\expandafter\@xs@addtomacro\expandafter\@xs@reserved@C\expandafter{\expandafter{\@xs@reserved@C}}%
			\@xs@read@reserved@D
		}%
		\def\@xs@atoccurfound{\let\@xs@reserved@C\@empty\@xs@read@reserved@D}%
		\@xs@read@reserved@D
		\expandafter\@xs@ReturnResult\expandafter{\number\decimalpart}{#3}%
		}%
}

% renvoie la position de l'occurrence #1 de la sous chaine #3 dans la chaine #2
\@xs@newmacro\StrPosition{1}{2}{1}{%
	\@xs@cutafteroccur{#2}{#3}{#1}%
	\let\@xs@reserved@E\groupID
	\ifx\@xs@reserved@C\@xs@reserved@D
		\@xs@ReturnResult{0}{#4}%
		\let\@xs@reserved@E\@empty
	\else
		\expandafter\@xs@StrLen\expandafter{\@xs@reserved@C}[\@xs@reserved@C]%
		\expandafter\@xs@ReturnResult\expandafter{\number\numexpr\@xs@reserved@C+1}{#4}%
	\fi
	\global\let\groupID\@xs@reserved@E
}

\def\comparestrict{\let\@xs@comparecoeff\@ne}

\def\comparenormal{\let\@xs@comparecoeff\z@}

\def\savecomparemode{\let\@xs@saved@comparecoeff\@xs@comparecoeff}

\def\restorecomparemode{\let\@xs@comparecoeff\@xs@saved@comparecoeff}

% Compare les deux arguments #1 et #2
% Renvoie 0 s'ils sont \'egaux et renvoie la
% position de la premiere unit\'e syntaxiques diff\'erente sinon
\@xs@newmacro*2\StrCompare{}{2}{1}{%
	\def\@xs@reserved@A{#1}%
	\def\@xs@reserved@B{#2}%
	\ifx\@xs@reserved@B\@xs@reserved@A
		\@xs@ReturnResult{0}{#3}%
	\else
		\def\@xs@next{\@xs@StrCompare@i{#1}{#2}{#3}}%
		\expandafter\@xs@next
	\fi
}

\long\def\@xs@StrCompare@i#1#2#3{%
	\def\@xs@StrCompare@iii##1{%
		\let\@xs@reserved@A\@empty
		\expandafter\@xs@ifempty\expandafter{\@xs@reserved@C}%
			{\def\@xs@reserved@A{*\@xs@comparecoeff}}%
			{\expandafter\@xs@ifempty\expandafter{\@xs@reserved@D}%
				{\def\@xs@reserved@A{*\@xs@comparecoeff}}
				\relax
			}%
		\def\@xs@next{%
			\expandafter\@xs@ReturnResult\expandafter
			{\number\numexpr##1\@xs@reserved@A\relax}{#3}%
		}%
	}%
	\def\@xs@StrCompare@ii##1{% ##1 est la position
		\expandafter\@xs@returnfirstsyntaxunit\expandafter{\@xs@reserved@C}\@xs@reserved@A
		\expandafter\@xs@returnfirstsyntaxunit\expandafter{\@xs@reserved@D}\@xs@reserved@B
		\ifx\@xs@reserved@B\@xs@reserved@A
			\expandafter\@xs@ifempty\expandafter{\@xs@reserved@A}%
				{\@xs@StrCompare@iii{##1}% les 2 unit\'es syntaxiques sont \'egales, on renvoie la position
				}
				{\def\@xs@next{\expandafter\@xs@StrCompare@ii\expandafter{\number\numexpr##1+1}}% les 2 unit\'es syntaxiques sont \'egales et non vides, on recommence
				\expandafter\@xs@removefirstsyntaxunit\expandafter{\@xs@reserved@C}\@xs@reserved@C
				\expandafter\@xs@removefirstsyntaxunit\expandafter{\@xs@reserved@D}\@xs@reserved@D
				}%
		\else% les 2 unit\'es syntaxiques sont diff\'erentes : on renvoie la position
			\@xs@StrCompare@iii{##1}%
		\fi
		\@xs@next
	}%
	\def\@xs@reserved@C{#1}\def\@xs@reserved@D{#2}%
	\@xs@StrCompare@ii1%
}

\@xs@newmacro\StrFindGroup{}{2}{1}{%
	\def\@xs@reserved@A{#2}\def\@xs@reserved@B{0}%
	\ifx\@xs@reserved@A\@xs@reserved@B
		\def\@xs@next{\@xs@ReturnResult{{#1}}{#3}}%
	\else
		\def\@xs@next{\@xs@StrFindGroup@i{#1}{#2}[#3]}%
	\fi
	\@xs@next
}

\long\def\@xs@StrFindGroup@i#1#2[#3]{%
	\def\@xs@StrFindGroup@ii{%
		\expandafter\@xs@ifempty\expandafter{\@xs@reserved@C}%
			{\def\@xs@next{\@xs@ReturnResult{}{#3}}% s'il ne reste plus rien, on renvoie vide
			}
			{\expandafter\@xs@returnfirstsyntaxunit\expandafter{\@xs@reserved@C}\@xs@reserved@D
			\ifx\bgroup\@xs@toks% si la 1\`ere unit\'e syntaxique est un groupe explicite
				\advance\decimalpart\@ne% on augmente le compteur
				\ifnum\decimalpart=\@xs@reserved@A% on est au groupe cherch\'e lors de la profondeur courante ?
					\ifx\@empty\@xs@reserved@B% on est \`a la profondeur maximale ?
						\def\@xs@next{\expandafter\@xs@ReturnResult\expandafter{\@xs@reserved@D}{#3}}% on renvoie ce groupe
					\else% sinon
						\expandafter\def\expandafter\@xs@next\expandafter{\expandafter\@xs@StrFindGroup@i\@xs@reserved@D}% on recommence avec ce groupe
						\expandafter\@xs@addtomacro\expandafter\@xs@next\expandafter{\expandafter{\@xs@reserved@B}[#3]}% et les profondeurs de recherche restantes
					\fi
				\else
					\expandafter\@xs@removefirstsyntaxunit\expandafter{\@xs@reserved@C}\@xs@reserved@C
					\let\@xs@next\@xs@StrFindGroup@ii
				\fi
			\else
				\expandafter\@xs@removefirstsyntaxunit\expandafter{\@xs@reserved@C}\@xs@reserved@C
				\let\@xs@next\@xs@StrFindGroup@ii
			\fi
			}%
		\@xs@next
	}%
	\@xs@extractgroupnumber{#2}\@xs@reserved@A\@xs@reserved@B
	\decimalpart\z@
	\ifnum\@xs@reserved@A>\z@\def\@xs@reserved@C{#1}\else\let\@xs@reserved@C\@empty\fi
	\@xs@StrFindGroup@ii
}

\def\@xs@extractgroupnumber#1#2#3{%
	\def\@xs@extractgroupnumber@i##1,##2\@xs@nil{\def#2{##1}\def#3{##2}}%
	\@xs@extractgroupnumber@i#1,\@xs@nil
	\ifx\@empty#3\else\@xs@extractgroupnumber@i#1\@xs@nil\fi
}

\def\expandingroups{\let\@xs@expandingroups\exploregroups}

\def\noexpandingroups{\let\@xs@expandingroups\noexploregroups}

\def\StrExpand{\@testopt{\@xs@StrExpand}{1}}
\long\def\@xs@StrExpand[#1]#2#3{%
	\begingroup
		\@xs@expandingroups
		\ifnum#1>\z@
			\integerpart#1\relax
			\decimalpart\z@
			\gdef\groupID{0}%
			\let\@xs@nestlevel\z@
			\def\@xs@atendofgroup{%
				\expandafter\endgroup
				\expandafter\@xs@addtomacro\expandafter\@xs@reserved@B\expandafter{\expandafter{\@xs@reserved@B}}%
				\@xs@read@reserved@C
			}%
			\def\@xs@atbegingroup{\let\@xs@reserved@B\@empty}%
			\def\@xs@atnextsyntaxunit{%
				\expandafter\expandafter\expandafter\@xs@addtomacro
				\expandafter\expandafter\expandafter\@xs@reserved@B
				\expandafter\expandafter\expandafter{\@xs@reserved@A}%
			}%
			\def\@xs@reserved@C{#2}%
			\@xs@StrExpand@i{#1}% Appel de la macro r\'ecursive
		\else
			\def\@xs@reserved@B{#2}%
		\fi
		\global\let\@xs@reserved@B\@xs@reserved@B
	\endgroup
	\let#3\@xs@reserved@B
	\global\let\groupID\@empty
}

\long\def\@xs@StrExpand@i#1{%
	\ifnum#1>\z@
		\let\@xs@reserved@B\@empty
		\@xs@read@reserved@C
		\let\@xs@reserved@C\@xs@reserved@B
		\def\@xs@reserved@A{\expandafter\@xs@StrExpand@i\expandafter{\number\numexpr#1-1}}%
	\else
		\let\@xs@reserved@A\relax
	\fi
	\@xs@reserved@A
}

\def\scancs{\@testopt{\@xs@scancs}{1}}
\long\def\@xs@scancs[#1]#2#3{%
	\@xs@StrExpand[#1]{#3}{#2}%
	\edef#2{\detokenize\expandafter{#2}}%
}

% Restaure le code de @
\catcode`\@=\CurrentAtCatcode\relax

% D\'efinit le d\'elimiteur verb
% d\'efinit les diff\'erents modes et les m\'emorise
\setverbdelim{|}%
\fullexpandarg\saveexpandmode
\comparenormal\savecomparemode
\noexploregroups\saveexploremode
\expandingroups

\endinput
%
% Historique
%------------------------------------------------------------------------------
%% v1.0    4/7/2008 : - Package achev\'e.
%          7/7/2008 : - Package publi\'e sur CTAN.
%------------------------------------------------------------------------------
% v1.1    15/8/2008
%   - Le package ne n\'ecessite plus LaTeX et est d\'esormais utilisable sous
%     Plain eTeX.
%   - Ajout des macros \IfInteger et \IfDecimal.
%     Pour cette derni\`ere, la partie enti\`ere et la partie d\'ecimale sont
%     r\'ecup\'erables dans les compteurs \integerpart et \decimalpart. Ce
%     qu'il reste apr\`es qu'un caract\`ere non attendu soit rencontr\'e est
%     dispo dans \@xs@remainafterdecimal
%   - Correction d'un bug dans \tokenize :
%        suppression de l'expansion avec \normalexpandarg pour \'eviter le bug
%        avec les lettres accentu\'ees dans l'exemple : \tokenize\aa{\'e\`u}
%   - Am\'elioration du code de la plupart des macros.
%   - \'Ecriture d'une routine \@xs@newmacro d\'efinissant les  macros.
%   - Suppression de tous les compteurs grace \`a \numexpr.
%   - Nouvelles fonctionnalit\'es internes dans :
%      \IfSubStr
%        ajout des chaines se trouvant avant et apr\`es l'occurrence test\'ee
%        dans les sc \@xs@behind@IfSubStr et \@xs@before@IfSubStr
%      \StrCount
%        a) ajout de ce qui se trouve apr\`es la derni\`ere occurrence dans la
%           sc \@xs@StrCount@afterlastocccur
%        b) la macro ne d\'epend plus d'aucune autre
%------------------------------------------------------------------------------
% v1.2    23/8/2008
%   - Am\'elioration du code de \StrMid \StrGobbleLeft, \StrLeft,
%     \StrGobbleRight et \StrRight \`a l'aide de la nouvelle macro
%     \@xs@StrSplit@i
%   - Nouveau test \IfStrEq qui compare 2 chaines
%   - Nouveau test \IfEq qui compare 2 chaines (ou 2 nombres si les 2 chaines
%     contiennent un nombre)
%   - Correction d'un bug dans \@xs@formatnumber. La macro n'envisageait pas le
%     cas o\`u son argument \'etait vide ce qui faiasait planter
%     \IfDecimal{}{V}{F} et \IfInteger{}{V}{F}
%------------------------------------------------------------------------------
% v1.3    21/9/2008
%  - Am\'elioration du code, mise en place de r\'ecursivit\'es terminales dans
%    toutes les macros r\'ecursives, simplification de \@xs@TestEqual en
%    utilisant \ifx
%  - Nouvelles macros \IfStrEqCase et \IfEqCase
%      Teste si une chaîne est \'egale \`a plusieurs cha\^ines donn\'ee et
%      ex\'ecute un code sp\'ecifique
%  - Cr\'eation de la macro publique \StrSplit
%      Coupe la chaine en 2 \`a une position donn\'ee
%  - Nouvelle macro \StrCompare
%      Compare 2 chaînes et renvoie la position du 1er caract\`ere diff\'erent.
%      Renvoie 0 si elles sont \'egales.
%------------------------------------------------------------------------------
% v1.3a   29/9/2008
%  - Correction d'un bug : \expandafter#1
%    dans les macros \IfSubStrBefore et \StrBetween qui buggue si #1 est un
%    nombre > 9 !
%  - R\'e \'ecriture de \@xs@returnfirstsyntaxunit et \@gobble qui
%    deviennent purement d\'eveloppables.
%  - R\'e \'ecriture de \StrLen
%  - Nouvelle macro interne \@xs@cutafteroccur
%      Coupe la chaine apr\`es la ni\`eme occurrence d'une sous chaine.
%  - Code am\'elior\'e de \StrMid, \StrLeft, \StrGobbleLeft, \StrRight,
%    \StrGobbleRight grace \`a \@xs@cutafteroccur
%------------------------------------------------------------------------------
% v1.4   4/11/2008
%  - xstring traite ses arguments par unit\'e syntaxique, et donc toutes les
%    macros internes ont \'et\'e r\'e-\'ecrites \`a cette fin.
%  - Le manuel a \'et\'e r\'e-\'ecrit, rendu plus clair et mieux structur\'e.
%  - Nouvelle fonctionnalit\'e : \expandarg
%      le 1er lex\`eme de chaque argument est 1-d\'evelopp\'e avant que la
%      macro ne soit appel\'ee. Si un argument comporte plus qu'1 lex\`eme,
%      il faut passer par \scancs*[1] pour 1-d\'evelopper toutes les unit\'es
%      syntaxiques.
%  - Correction de 2 erreurs dans le manuel anglais :  0 est la valeur par
%    d\'efaut dans \StrSubstitute et \StrDel
%  - Coh\'erence dans les nom des s\'equences de contr\^oles temporaires qui
%    sont d\'esormais : \@xs@reserved@A \`a \@xs@reserved@E
%------------------------------------------------------------------------------
% v1.4a  12/11/2008
%  - Correction d'un bug dans \@xs@StrSplit dans le cas o\`u l'argument
%     num\'erique est <1 : la 1ere coupure doit être vide et le seconde doit
%     être \'egale \`a l'argument.
%------------------------------------------------------------------------------
% v1.4b  16/11/2008
%  - Correction de 2 bugs dans \@xs@returnfirstsyntaxunit
%------------------------------------------------------------------------------
% v1.5   31/12/2008
%  - Nouvelles fonctionnalit\'es dans la recherche de groupes par un
%    identificateur caract\'erisant leur imbrication. Impl\'ementation de :
%      1) \StrFindGroup
%      2) \groupID
%  - Ajout des nouvelles macros g\'en\'eriques \@xs@read@reserved@C et
%    \@xs@read@reserved@D
%  - Correction d'un bug dans \StrSplit : \StrSplit{{a}bc}{1}[\truc] faisait
%    que \truc contenait «a{a}».
%  - Correction d'un bug dans \@xs@newmacro : l'argument optionnel \'etait
%    d\'etok\'enis\'e \`a tort dans les macros \'etoil\'ees.
%  - Dans \@xs@newmacro, on doit choisir le num\'ero du dernier argument devant
%    être d\'etok\'enis\'e dans le cas d'une macro \'etoil\'ee (sachant que
%     l'argument optionnel ne l'est jamais)
%------------------------------------------------------------------------------
% v1.5a  09/2/2009
%  - Nouvelle macro \StrExpand, sensible au mode d'exploration. La macro est
%    indépendante du mode d'exploration et dispose de ses propres commandes
%    d'exploration : \expandingroups et \noexpandingroups
%  - R\'e-\'ecriture de \scancs, qui devient triviale puisque d\'ependant de
%    \StrExpand. Il n'y a plus de version \'étoil\'ee et d\'esormais, \scancs
%    est sensible au mode d'exploration.
%  - Correction d'un bug dans \IfInteger : \IfInteger{2.0}{V}{F} affichait F
%  - Correction de bugs : mettre \let\@xs@reserved@E\relax avant d'appeler
%    \@xs@StrSplit@i
%  - Suppression des messages d'erreur, de warning et d'info. xstring devient
%    donc silencieux ;)
%  - Passage de 3 \'a 2 compteurs
%  - Le manuel est d\'esormais plus color\'e, en esp\'erant qu'il sera un peu
%    plus agr\'eable \`a lire !
%------------------------------------------------------------------------------
% v1.5b  13/3/2009
%  - Modification d'un grand nombre de macros de telle sorte que vraiment
%    n'importe quel code peut d\'esormais \^etre plac\'e dans les arguments des
%    macros (comme par exemple des \if, \ifx ou autre, m\^eme sans leur \fi, ce
%    qui n'\'etait pas possible jusqu'alors)
%------------------------------------------------------------------------------
% v1.5c  05/6/2009
%  - Correction d'un bug dans \IfBeginWith : il faut appeler
%    \@xs@returnfirstsyntaxunit pour l'argument #2 puis pour l'argument #1 et
%    non pas l'inverse pour que \@xs@toks reflète ce qui se trouve dans #1
%  - Correction d'un bug dans \@xs@returnfirstsyntaxunit au cas où #1 commence
%    par un espace : mettre \@xs@toks à 0
%  - Correction d'un bug dans \@xs@returnfirstsyntaxunit :
%    \@xs@returnfirstsyntaxunit{{}}\truc
%    faisait que \truc se développait en un espace.
%  - Correction d'un bug dans \@xs@removefirstsyntaxunit où le cas
%    \@xs@removefirstsyntaxunit{ {b}}\truc
%    donnait \truc se développant en «b» au lieu de «{b}».
%------------------------------------------------------------------------------
% v1.5d  28/3/2010
%  - Correction d'un bug dans \IfInteger : \@xs@afterinteger ne donnait pas le
%    bon r\'esultat. Par exemple avec 1.23 il renvoyait {} au lieu de {.23}
%  - \xs@formatnumber ne detokenize plus ses arguments (que m'est-il pass\'e
%    par la t\^ete pour que je code un truc comme ca ?), et donc
%    \@xs@afterinteger et \@xs@afterdecimal ont des tokens ayant le bon
%    catcode
%------------------------------------------------------------------------------
% v1.6  24/10/2012
%  - Correction d'un bug dans \@xs@cutafteroccur au cas o\`u #3<1
%  - \IfDecimal ne \detokenize plus l'argument.
%  - Nouvelle version de \tokenize qui enlevait les accolades si son argument
%    \'etait entre accolades
%  - Nouvelle macro \StrCut
%  - \@xs@cutafteroccur@i ne produit plus d'espace indésirable si #3 était
%    évalué par \numexpr
%------------------------------------------------------------------------------
% v1.7  13/1/2013
%  - Correction d'un manque de rigueur dans les assignations concernant
%    \groupID : elles sont toutes globales d\'esormais
%  - Les macros de xstring sont \long
%------------------------------------------------------------------------------
% v1.7a 28/2/2013
%  - Correction d'un espace parasite dans \@xs@removefirstsyntaxunit
%------------------------------------------------------------------------------
% v1.7b 29/7/2013
%  - La date contenue dans \xstringdate est désormais conforme