\documentclass[11pt,a4paper]{article}
% Presumably we will have a modified template header for the final papers
% For now we hard code things below

\usepackage{fancyhdr}
\usepackage{parskip}
\usepackage[usenames,dvipsnames]{color}
\usepackage[colorlinks=true, linkcolor=BrickRed, citecolor=Blue, urlcolor=Blue, filecolor=Blue]{hyperref}
\usepackage{graphicx}
\usepackage{xcolor}
\usepackage{amssymb, amsmath}
\usepackage{lmodern}
\usepackage[T1]{fontenc}
\usepackage{lineno}
\newcommand{\tab}{${}$\ \ \ \ }
%\pagestyle{fancy}

\newcommand{\mail}[1]{\href{mailto:#1}{#1}}

\title{Comparison of statistical scanning methods with ScannerBit: a GAMBIT module for parameter scanning}

\author{GAMBIT Scanner Working Group}

\begin{document}

\maketitle

\begin{abstract}
blah
\end{abstract}

\section{Introduction}
Context of GAMBIT

\section{Background}
\subsection{Statistics}
brief background on frequentist profile likelihood and bayesian methods 
\subsection{Scanning methods}
background on scanning methods in global fits (mastercode, superbayes, fittino, cosmomc, etc)

\section{Package description}
general, including description of "purposes"
\subsection{Scanner plugins}
including plugin design and handles for interface with ScannerBit + interface to external libs (cmake + python system)
\subsection{Target plugins}

\section{Implemented Scanners}
\subsection{Toy scanners: toy\_mcmc, random, grid}
\subsection{Nested sampling}
\subsection{Differential evolution}
\subsection{T-walk}

\section{Scanner comparison and examples}
\begin{itemize}
\item test target function performance of each scanner
\item implementation examples in gambit and standalone
\end{itemize}

\section{Conclusions}

\section{Acknowledgements}

\appendix

\section{Scanner Example}

Below is code that declares the scanner plugin ``random'', version ``1.0.0-example''.
This scanner enters ``\texttt{number}'' random points in the functor corresponding to
the purpose ``\texttt{Likelihood}''.

\begin{linenumbers}
\texttt{\#include "scanner\_plugin.hpp"\\
\\
scanner\_plugin(random, version(1, 0, 0, example))\\
\{\\
\tab reqd\_inifile\_entries("number");\\
%\tab reqd\_libraries("my\_cool\_library");\\
%\tab reqd\_include\_paths("my\_cool\_header.h");\\
\tab scan\_ptr <double (const std::vector<double> \&)> loglike;
\\
\tab int num, dim;\\
\\
\tab plugin\_constructor\\
\tab \{\\
\tab \tab likelihood = get\_functor("Likelihood");\\
\tab \tab num = get\_inifile\_value<int>("number");\\
\tab \tab dim = get\_dimension();\\
\tab \}\\
\\
\tab int plugin\_main(void)\\
\tab \{\\
\tab \tab std::vector<double> a(dim);\\
\tab \tab for (int j = 0; j < num; j++)\\
\tab \tab \{\\
\tab \tab \tab for (int i = 0; i < dim; i++)\\
\tab \tab \tab \{\\
\tab \tab \tab \tab a[i] = Gambit::Random::draw();\\
\tab \tab \tab \}\\
\tab \tab \tab loglike(a);\\
\tab \tab \}\\
\\
\tab \tab return 0;\\
\tab \}\\
\\
\tab plugin\_deconstructor\\
\tab \{\\
\tab \tab std::cout << "no more plugin" << std::endl;\\
\tab \}\\
\}}
\end{linenumbers}

\section{Objective Example}

\begin{linenumbers}
\texttt{objective\_plugin(simple\_gaussian, version(1,0,0))\\
\{\\
\tab double plugin\_main (const std::vector<double> \&vec)\\
\tab \{\\
\tab \tab std::vector<double> params = prior\_transform(vec);\\
\\
\tab \tab return 0;\\
\tab \}\\
\}}
\end{linenumbers}

\section{Input file options}

\section{Plugin Commands}
Here are the different macros and functions that can be used:

\texttt{scanner\_plugin(plugin\_name, version) \{...\}}\\
-- declared a plugin with the name "plugin\_name".
   
\texttt{objective\_plugin(plugin\_name, version) \{...\}}
-- same as above, except declares an "objective" plugin.
   Used as test functions and prior plugins.
   
\texttt{version(majar, minor, patch, release)}
-- declares the version number to be "major.minor.patch-release"
   major, minor, and patch must be integers.  release is optional.
   
\texttt{ret plugin\_main(args... params) \{...\}}
-- The function that is called when the plugin is ran.  For the
   scanner, it must be of the form "int plugin\_name(void)"
   
\texttt{plugin\_constructor \{...\}}
-- This function will be ran when the plugin to loaded.

\texttt{plugin\_deconstructor \{...\}}
-- This function will be ran when the plugin to closed.

\texttt{get\_inifile\_value<ret\_type>("tag", default)}
-- Gets the inifile value corresponding to the tag "tag".
   If "tag" is not specified, then the results defaults
   to "default".  The default entry is optional. If
   the default entry and the "tag" infile entry is not
   specified, ScannerBit will throw an error.
   
\texttt{get\_dimension()}
-- Gets the dimension of the unit hypercube being explored.

\texttt{get\_purpose("purpose")}
-- gets the functor corresponding to the purpose "purpose"

\texttt{scan\_ptr<ret (args...)>}
-- a container functor of the form "ret scan\_ptr(args ...)".
   Used to contain the output of "get\_functor".
   
\texttt{set\_dimension(...)}
-- For objective plugins that will be used as priors.
   Set the hypercube dimension that will be operator
   over by the prior.
   
\texttt{reqd\_inifile\_entries(...)}
-- Tells ScannerBit that there entries must be in the inifile
   in order to load the plugin
   
\texttt{reqd\_libraries(...)}
-- Tells ScannerBit to search for and link these libraries

\texttt{reqd\_headers(...)}
-- Tells ScannerBit that these headers are required to exist
   in order for the plugin to compile.

\section{Dependencies}

\end{document}
