Despite decades of collider searches for physics beyond the Standard Model (BSM), it remains the case that we lack an unambiguous discovery of such physics. The many null results from the Large Hadron Collider (LHC) and other experiments allow us to constrain, to various degrees, the parameter spaces of any extension to the SM, including ``bottom-up'' effective theories of dark matter, supersymmetric theories, theories with extra space dimensions and composite Higgs or Technicolour models. Since a generic theory of BSM physics will have observable consequences in a large range of experiments, it is particularly important to combine collider exclusions with other experiments in a statistically rigorous way if one is to draw sound conclusions on the viability of a candidate theory.

It remains difficult to apply collider results to a generic theory of BSM physics. Searches for new particles at the LHC are typically presented either in particular planes of a GUT scale physics hypothesis (e.g the constrained minimal supersymmetric model), or in simplified models that only strictly apply to a very small volume of the total allowed space of particle masses and branching ratios. The computational expense of simulating signal processes for hundreds of thousands of points in a candidate model prevents an extended treatment. In addition, some LEP results remain state of the art, and are not always rigorously applied in the literature. Finally, the discovery and subsequent observations of a SM-like Higgs boson by the ATLAS and CMS experiments in 2012 provides tight constraints on variations in the Higgs branching ratios that must be included in any thorough exploration of a BSM physics model~\cite{Aad:2012tfa,Chatrchyan:2012xdj}.

Partial solutions to each of these issues exist, but there is as yet no comprehensive tool that tackles all of them. The package \tt SmodelS \rm applies constraints to supersymmetric models based on a combination of simplified model results~\cite{Kraml:2013mwa}. \tt Fastlim \rm provides similar functionality for SUSY models, but is extendable (in principle) to non-SUSY models through the use of user-supplied efficiency tables~\cite{Papucci:2014rja}. Both of these tools will provide limits that are much more conservative than a more rigorous calculation, due to the limitations of simplified models.\tt CheckMATE \rm provides a customised version of the \tt DELPHES \rm detector simulation, an event analysis framework and a list of ATLAS and CMS analyses which can be used to apply LHC limits to candidate BSM models provided that the user supplies production cross-sections and a sample of Monte Carlo events~\cite{Drees:2013wra,deFavereau:2013fsa,Ovyn:2009tx}.  To the best of our knowledge, no general purpose tool exists to apply LEP BSM search limits limits, although many theorists have implemented their own local codes over the years. On the Higgs side, the community is well served by the packages \tt HiggsBounds \rm and \tt HiggsSignals \rm~\cite{Bechtle:2008jh,Bechtle:2011sb,Bechtle:2013wla}. 

We present a new software code, \tt ColliderBit \rm for the application of high energy collider constraints to BSM physics theories. The code is designed within the \tt GAMBIT \rm framework, and thus offers seamless integration with other packages that provide a statistical fitting framework, and also the ability to impose constraints from electroweak precision data, flavour physics and a large range of astrophysical observations. Furthermore, the code is modular in design, allowing the user to easily swap components, add new collider analyses or provide interfaces to standard particle physics tools. For LHC physics, we use a combination of parallelised Monte Carlo simulation and fast detector simulation to recast LHC limits without the approximations of the simplified model approach. The package is supplied with a selection of the most important ATLAS and CMS analyses (focussing initially on supersymmetric and dark matter searches), and contains interfaces to the \tt Pythia 8 \rm MC event generator and \tt DELPHES \rm detector simulation, in addition to a customised detector simulation based on four-vector smearing that is found to give comparable results to \tt DELPHES \rm with a reduced CPU overhead. We supply custom routines for re-evaluating LEP limits on supersymmetric particle production, and include an interface to \tt HiggsBounds \rm for the calculation of Higgs observables.

This paper serves as both a description of the physics and design strategy for ColliderBit, and a user manual for the first code release. In Section~\ref{sec:quickstart}, we provide a quick start guide for users keen to compile and use the software out of the box. Section~\ref{sec:background} briefly describes the physics background necessary for describing our design strategy. The \tt ColliderBit \rm user interface is outlined in Section~\ref{sec:interface} before we give a detailed explanation of the code itself in Section~\ref{sec:code}. Finally, Section~\ref{sec:examples} contains a detailed description of two use cases; the application of collider constraints to a) the minimal supersymmetric standard model (MSSM), and b) a toy model which is not currently included in the \tt ColliderBit code\rm. The second of these examples demonstrates the fleixibility of \tt ColliderBit \rm in tackling generic theories supplied by the user (and we note that we here rely heavily on existing interfaces for automatic matrix element generation).

