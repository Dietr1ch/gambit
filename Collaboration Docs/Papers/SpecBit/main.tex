\documentclass[11pt,a4paper]{article}
% Presumably we will have a modified template header for the final papers
% For now we hard code things below

\usepackage{fancyhdr}
\usepackage{parskip}
\usepackage[usenames,dvipsnames]{color}
\usepackage[colorlinks=true, linkcolor=BrickRed, citecolor=Blue, urlcolor=Blue, filecolor=Blue]{hyperref}
\usepackage{graphicx}
\usepackage{xcolor}
\usepackage{amssymb, amsmath}
%\pagestyle{fancy}

\newcommand{\mail}[1]{\href{mailto:#1}{#1}}

\title{SpecBit: A GAMBIT module for the linking and running spectrum genetaors and decay codes}

\author{The GAMBIT Collaboration: Alphabetical author list} % will presumably have a standard author list

\begin{document}

\maketitle

\begin{abstract}
Write in exactly same style as CollierBit if possible.
%% We present a new code for the calculation of high energy collider observables, given a generic theory of beyond the Standard Model physics. Describe novel features of HEColliderBit (parallelised MC generation, fast SUSY cross-section calculations, generic interface to BSM models, lots of LHC analyses). Mention link to GAMBIT framework, but emphasise that the package presents a standalone solution to the problem of applying LHC constraints to new physics theories.
\end{abstract}

\tableofcontents


\section{Introduction}
motivation, literature review, summary of advances in this paper
%% Wealth of recent LHC data provides strong constraints on new physics models. It remains difficult to apply LHC constraints to generic physics models in a rigorous fashion. Various attempts have been made to automate the process (SmodelS, CheckMATE, etc). No attempt has attacked the problem in a fully rigorous way, and neither have any existing attempts been integrated into a fully general framework for statistical fits of generic BSM theories. 

%% We present a new, fully general solution based on parallelised Monte Carlo simulation, fast detector simulation and fast cross-section calculations for SUSY processes. Part of the Global and Module BSM Inference Tool, which will include, etc. 

%% This paper is organised as follows. In Section ?? we..., etc. 

\section{Quick Start}

\section{Physics background}
MSSM Superpotential etc to lay out conventions
\section{Overview of physics models supported}
MSSM will be most common, just list ones that are fully supported without user modification?
\begin{itemize} 
\item MSSM
  \begin{itemize}
  \item CMSSM  
  \item NUHM 
  \item MSSM-25
  \end{itemize} 
\item NMSSM
  \begin{itemize}
  \item CNMSSM  
  \item NUHNMSSM 
  \item NMSSM-30
  \end{itemize}
\item other models?
\end{itemize}  

\section{User Interface}

\subsection{MSSM setup} 
         default codes 
         (may not want this but softsusy and FS at least 
                         are special to Gambit at some level) 

         codes already implemented

         addding new codes: more advanced options if user adds own decay code
                            or FeynHiggs alternative
 
\subsection{NMSSM setup}
    quick repeat of MSSM but with MSSM specifics

\subsection{Other models}
    SUSY

    Non-SUSY


Most of this list from ColliderBit actually applies and should be in stuff above or just below them.
\begin{itemize}
\item Describe interface to GAMBIT framework.
\item Describe standalone interface (how to configure and run the package, what the defaults are).
\item Describe the interface to the models that we include and how to change the input parameters of those models.
\item Explain how to interface with a new model.
\item Explain how to run code in single core or multicore form (i.e using the standard OpenMP commands).
\item Describe parameters that can be varied in yaml file.
\end{itemize}



\section{Code description}
horrible horrible details


%%\subsection{User interface}


\section{Examples}
Give some tutorial-style examples for common use cases.

\subsection{MSSM example}
MSSM run even needed?

adding new MSSM generator (can just use FS example)
\subsection{NMSSM example}
NMSSM run even needed?

adding new NMSSM generator (can just use FS example)
\subsection{exotic SUSY with own tools example - E6SSM?}
May not be possible in time

\section{Conclusions}

\section{Acknowledgements}

\section{Appendix: ?}

\end{document}
