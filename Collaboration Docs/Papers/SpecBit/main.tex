\documentclass[11pt,a4paper]{article}
% Presumably we will have a modified template header for the final papers
% For now we hard code things below

\usepackage{fancyhdr}
\usepackage{parskip}
\usepackage[usenames,dvipsnames]{color}
\usepackage[colorlinks=true, linkcolor=BrickRed, citecolor=Blue, urlcolor=Blue, filecolor=Blue]{hyperref}
\usepackage{graphicx}
\usepackage{xcolor}
\usepackage{amssymb, amsmath}
%\pagestyle{fancy}

\newcommand{\mail}[1]{\href{mailto:#1}{#1}}

\title{SpecBit: GAMBIT modules for linking and running spectrum generators, decay codes and EWPO calculators}

\author{The GAMBIT Collaboration: Alphabetical author list} % will presumably have a standard author list

\begin{document}

\maketitle

\begin{abstract}
We present a new computer code which can link all publicly available spectrum generators, decay codes and electroweak precision observable calculators together 
% without the need of an uniformly agreed 
% without writing to any files internally 
[this may be too ambitious].
Describe novel features, FeynHiggs linking, easy to replace any code. Mention link to GAMBIT framework, but emphasise that the package presents a standalone solution to the problem of linking these codes together and maybe physics problem of matching too FeynHiggs.
\end{abstract}

\tableofcontents


\section{Introduction}

motivation, literature review, summary of advances in this paper

%% Wealth of recent LHC data provides strong constraints on new physics models. It remains difficult to apply LHC constraints to generic physics models in a rigorous fashion. Various attempts have been made to automate the process (SmodelS, CheckMATE, etc). No attempt has attacked the problem in a fully rigorous way, and neither have any existing attempts been integrated into a fully general framework for statistical fits of generic BSM theories. 

%% We present a new, fully general solution based on parallelised Monte Carlo simulation, fast detector simulation and fast cross-section calculations for SUSY processes. Part of the Global and Module BSM Inference Tool, which will include, etc. 

%% This paper is organised as follows. In Section ?? we..., etc. 

Describe interface to GAMBIT framework. (technical details toward the end of the paper)

\section{Quick Start Guide}


\section{Physics background}


\subsection{general intro}

phenomenology: from Lagrangian to observables

MSSM superpotential etc to lay out parameters and conventions (a la SLHA 2)

NMSSM superpotential etc.

other models:

- scalar singlet

- E6SSM


\subsection{SpecBit}

RGEs: statement of purpose

SUSY specific and non-SUSY examples:

- vacuum stability, gauge coupling unification, Yukawa unification

spectrum generators for SUSY and non-SUSY models
 
 
\subsection{DecayBit}

purpose: 

- quantum corrections to SM Higgs decays from new physics

- decays of BSM particles (MSSM, Z', ...)

role for LHC: connecting production of exotic particles to observable final states

role for dark matter???: decaying DM


\subsection{EWPOBit}

purpose: quantum corrections to SM observables from new physics

- $m_h$, $m_W$, $\sin(\theta_W)$, ...


\section{Overview of physics models supported}

MSSM will be most common, just list ones that are fully supported without user modification?

\subsection{SpecBit}

models already implemented: table with RGE codes references?

\begin{itemize} 
\item MSSM
  \begin{itemize}
  \item CMSSM 
  \item NUHM 
  \item MSSM-25
  \end{itemize} 
\item NMSSM
  \begin{itemize}
  \item CNMSSM  
  \item NUHNMSSM 
  \item NMSSM-30
  \end{itemize}
\item other models?
  \begin{itemize}
  \item scalar singlet
  \item E6MSSM  
  \end{itemize}
\end{itemize}  

interfaced spectrum generators (aspirational list)
         
generator specific BC options for SUSY models

\subsection{DecayBit}

same structure as SpecBit

\subsection{EWPOBit}

same struct as SpecBit


\section{User Interface}

In this section we describe the standalone interface, how to configure and run the package, what the defaults are.

Describe the interface to the models that we include and how to change the input parameters of those models.

Describe parameters that can be varied in yaml file. [included above]

general YAML blurb???

\subsection{General settings}

Explain how to run code in single core or multicore form (i.e using the standard OpenMP commands).
- multi-threading is default option [check this!]

\subsection{SpecBit}

\subsubsection{Scalar singlet}

explaining YAML instructions 

- input parameter list and values (as in FlexibleSUSY manual)

[along the line of the MSSM below]

\subsubsection{MSSM}

- explaining YAML instructions 

- input parameter list and values (as in FlexibleSUSY manual)

- default codes 
(may not want this but softsusy and FS at least 
are special to Gambit at some level) 

\subsubsection{NMSSM}

    quick repeat of MSSM but with MSSM specifics

\subsubsection{...}

\subsection{DecayBit}

interfacing to the decays

\subsubsection{Scalar singlet}

\subsubsection{MSSM}

describe decay table and give MSSM examples

\subsubsection{NMSSM}
\subsubsection{...}

- DecayTable

\subsection{EWPOBit}
\subsubsection{Scalar singlet}
\subsubsection{MSSM}
\subsubsection{NMSSM}
\subsubsection{...}
 

\section{Implementing new models and/or backends}

Explain how to interface with a new model.

\subsection{SpecBit}

- instructions for getters and setters and RGE running
$get(Pars:Pole\_Mass...)$

- specify string names and PDG codes

implementing a new model: equivalent of MSSMSpec.hpp


\subsection{DecayBit}

adding new codes: more advanced options if user adds own decay code or FeynHiggs alternative

\subsection{EWPOBit}



\section{Code description}

horrible horrible details:

code structure

\subsection{SpecBit}

- Spectrum, subspectrum object (structure based on headers?)

- simplified diagrams 

\subsection{DecayBit}


\subsection{EWPOBit}

\subsection{All together now}

- SLHA helpers

%%\subsection{User interface}


\section{Examples}
Give some tutorial-style examples for common use cases.

\subsection{MSSM example}
MSSM run even needed?

adding new MSSM generator (can just use FS example)
\subsection{NMSSM example}
NMSSM run even needed?

adding new NMSSM generator (can just use FS example)
\subsection{exotic SUSY with own tools example - E6SSM?}
May not be possible in time

\section{Outlook}

\section{Conclusions}
Gambit SpecBit,DeacyBayBit and EWPOBit will rule the world.  the rest of gambit is kind of useful too.
\section{Acknowledgements}

\end{document}
