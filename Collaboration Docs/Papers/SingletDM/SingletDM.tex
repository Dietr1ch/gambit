%%%%%%%%%%%%%%%%%%%%%%% file template.tex %%%%%%%%%%%%%%%%%%%%%%%%%
%
% This is a  template file for the LaTeX package SVJour3 width change file svepjc3.clo
% for Springer journal:
% The European Physical Journal C
%
% Copy it to a new file with a new name and use it as the basis
% for your article. Delete % signs as needed.
%
% This template includes a few options for different layouts and
% content for various journals. Please consult a previous issue of
% your journal as needed.
%
%%%%%%%%%%%%%%%%%%%%%%%%%%%%%%%%%%%%%%%%%%%%%%%%%%%%%%%%%%%%%%%%%%%
%
% First comes an example EPS file -- just ignore it and
% proceed on the \documentclass line
% your LaTeX will extract the file if required
%\begin{filecontents*}{example.eps}
%!PS-Adobe-3.0 EPSF-3.0
%%BoundingBox: 19 19 221 221
%%CreationDate: Mon Sep 29 1997
%%Creator: programmed by hand (JK)
%%EndComments
%gsave
%newpath
%  20 20 moveto
%  20 220 lineto
%  220 220 lineto
%  220 20 lineto
%closepath
%2 setlinewidth
%gsave
%  .4 setgray fill
%grestore
%stroke
%grestore
%\end{filecontents*}
%
\RequirePackage{fix-cm}
%
\documentclass[twocolumn,epjc3]{svjour3}  
%
\smartqed  % flush right qed marks, e.g. at end of proof
%
\RequirePackage{graphicx}
%
% \RequirePackage{mathptmx}      % use Times fonts if available on your TeX system
%
% insert here the call for the packages your document requires
\RequirePackage{latexsym}
%\RequirePackage[numbers,sort&compress]{natbib}
%\RequirePackage[colorlinks,citecolor=blue,urlcolor=blue,linkcolor=blue]{hyperref}
% etc.
%
% please place your own definitions here and don't use \def but
% \newcommand{}{}
%
\journalname{Eur. Phys. J. C}
%
\begin{document}

\title{Singlet Dark Matter with Gambit}

%\subtitle{Do you have a subtitle?\\ If so, write it here}

\titlerunning{Singlet Dark Matter with Gambit}        % if too long for running head

\author{The GAMBIT collaboration: Torsten Bringmann, Jan Conrad, Jonathan M.~Cornell, Lars A.~Dal, Joakim Edsj\"o, Miguel Pato, Antje Putze, Christopher Savage, Pat Scott, Christoph Weniger + XXX\thanksref{e1,addr1,addr2}
%        \and
%        Second Author\thanksref{e2,addr2,addr3} %etc.
}

\thankstext{e1}{e-mail: resistanceisfutile@gambit.com}

%\authorrunning{Short form of author list} % if too long for running head

\institute{First address \label{addr1}
           \and
           Second address \label{addr2}
%           \and
%           \emph{Present Address:} if needed\label{addr3}
}

\date{\today}


\newcommand{\DB}{\mbox{\sf DarkBit}}
\newcommand{\GB}{\mbox{\sf GAMBIT}}


\maketitle

\begin{abstract}

Goal: 'compact' paper with emphasis on results
Essence: the most up-to-date and comprehensive statistical analysis of the model

%\keywords{First keyword \and Second keyword \and More}
% \PACS{PACS code1 \and PACS code2 \and more}
% \subclass{MSC code1 \and MSC code2 \and more}
\end{abstract}

%%%%%%%%%%%%%%%%%%%%%%%%%%%%%%%%%%%%%%%%%%%%%%%%%%%%%%%
%%%%%%%%%%%%%%%%%%%%%%%%%%%%%%%%%%%%%%%%%%%%%%%%%%%%%%%
\section{Introduction}
\label{intro}

\begin{itemize}
  \item motivation
  \item literature review
  \item summary of advances in this paper
\end{itemize}

%%%%%%%%%%%%%%%%%%%%%%%%%%%%%%%%%%%%%%%%%%%%%%%%%%%%%%%
%%%%%%%%%%%%%%%%%%%%%%%%%%%%%%%%%%%%%%%%%%%%%%%%%%%%%%%
\section{Physics framework}
\label{phys}


\begin{itemize}
           \item Lagrangian, parameters, etc. for the particle physics model
           \item further major physics assumptions (cosmology, astrophysics, etc.)
\end{itemize}



%%%%%%%%%%%%%%%%%%%%%%%%%%%%%%%%%%%%%%%%%%%%%%%%%%%%%%%
%%%%%%%%%%%%%%%%%%%%%%%%%%%%%%%%%%%%%%%%%%%%%%%%%%%%%%%
\section{Likelihood function}
\label{lnL}

In this section:
\begin{itemize}
  \item table of observables including name of observable, type of constrain,
    experimental value(s)/errors [references], theory errors
  \item description of prediction calculations
  \item Priors
\end{itemize}

Include for sure:
\begin{itemize}
  \item CMB
  \item Fermi LAT dwarfs pass 8
  \item Fermi LAT line limits
  \item XENON-100
  \item LUX
  \item HESS-I
  \item Higgs invisible
  \item Planck RD
\end{itemize}

Future projections:
\begin{itemize}
  \item CTA
  \item GAMMA-400 (?)
  \item HESS-II
  \item ILC
\end{itemize}

Nuisance parameters:
\begin{itemize}
  \item Higgs mass (ask Martin)
  \item Nucleon couplings (ask Chris)
  \item Local DM density
  \item GC DM density
\end{itemize}

Maybe:
\begin{itemize}
  \item Mono-X (ask Martin)
\end{itemize}

Things that don't matter (but are still worth mentioning)
\begin{itemize}
  \item CR anti-protons
  \item CR positrons
  \item Radio limits
  \item IceCube
\end{itemize}



%%%%%%%%%%%%%%%%%%%%%%%%%%%%%%%%%%%%%%%%%%%%%%%%%%%%%%%
%%%%%%%%%%%%%%%%%%%%%%%%%%%%%%%%%%%%%%%%%%%%%%%%%%%%%%%
\section{Current status}
\label{examples}

In this section:
\begin{itemize}
  \item Frequentist: profile likelihoods, p values
  \item Bayesian: marginalized posteriors, evidences (odds?)
  \item Discussion: comparison to existing results, new results, future prospects
\end{itemize}

Bayesian priors:
\begin{itemize}
  \item log-priors
  \item specify parameter range
\end{itemize}

Frequentist:
\begin{itemize}
  \item contours
  \item can one use p-values to discuss the results
\end{itemize}


%%%%%%%%%%%%%%%%%%%%%%%%%%%%%%%%%%%%%%%%%%%%%%%%%%%%%%%
%%%%%%%%%%%%%%%%%%%%%%%%%%%%%%%%%%%%%%%%%%%%%%%%%%%%%%%
\section{Future prospects}
\label{examples}

In this section:
\begin{itemize}
  \item Frequentist: profile likelihoods, p values
  \item Bayesian: marginalized posteriors, evidences (odds?)
  \item Discussion: comparison to existing results, new results, future prospects
\end{itemize}

Bayesian priors:
\begin{itemize}
  \item log-priors
  \item specify parameter range
\end{itemize}

Frequentist:
\begin{itemize}
  \item contours
  \item can one use p-values to discuss the results
\end{itemize}

%%%%%%%%%%%%%%%%%%%%%%%%%%%%%%%%%%%%%%%%%%%%%%%%%%%%%%%
%%%%%%%%%%%%%%%%%%%%%%%%%%%%%%%%%%%%%%%%%%%%%%%%%%%%%%%
\section{Conclusions}
\label{conc}

...



\begin{acknowledgements}
...
\end{acknowledgements}



% BibTeX users please use one of
%\bibliographystyle{spbasic}      % basic style, author-year citations
%\bibliographystyle{spmpsci}      % mathematics and physical sciences
%\bibliographystyle{spphys}       % APS-like style for physics
%\bibliography{}   % name your BibTeX data base

% Non-BibTeX users please use
\begin{thebibliography}{}
%
% and use \bibitem to create references. Consult the Instructions
% for authors for reference list style.
%
\bibitem{RefJ}
% Format for Journal Reference
Author, Article title, Journal, Volume, page numbers (year)
% Format for books
\bibitem{RefB}
Author, Book title, page numbers. Publisher, place (year)
% etc
\end{thebibliography}

\end{document}
% end of file template.tex

