%%%%%%%%%%%%%%%%%%%%%%% file template.tex %%%%%%%%%%%%%%%%%%%%%%%%%
%
% This is a  template file for the LaTeX package SVJour3 width change file svepjc3.clo
% for Springer journal:
% The European Physical Journal C
%
% Copy it to a new file with a new name and use it as the basis
% for your article. Delete % signs as needed.
%
% This template includes a few options for different layouts and
% content for various journals. Please consult a previous issue of
% your journal as needed.
%
%%%%%%%%%%%%%%%%%%%%%%%%%%%%%%%%%%%%%%%%%%%%%%%%%%%%%%%%%%%%%%%%%%%
%
% First comes an example EPS file -- just ignore it and
% proceed on the \documentclass line
% your LaTeX will extract the file if required
%\begin{filecontents*}{example.eps}
%!PS-Adobe-3.0 EPSF-3.0
%%BoundingBox: 19 19 221 221
%%CreationDate: Mon Sep 29 1997
%%Creator: programmed by hand (JK)
%%EndComments
%gsave
%newpath
%  20 20 moveto
%  20 220 lineto
%  220 220 lineto
%  220 20 lineto
%closepath
%2 setlinewidth
%gsave
%  .4 setgray fill
%grestore
%stroke
%grestore
%\end{filecontents*}
%
\RequirePackage{fix-cm}
%
\documentclass[twocolumn,epjc3]{svjour3}  
%
\smartqed  % flush right qed marks, e.g. at end of proof
%
\RequirePackage{graphicx}
%
% \RequirePackage{mathptmx}      % use Times fonts if available on your TeX system
%
% insert here the call for the packages your document requires
\RequirePackage{latexsym}
%\RequirePackage[numbers,sort&compress]{natbib}
%\RequirePackage[colorlinks,citecolor=blue,urlcolor=blue,linkcolor=blue]{hyperref}
% etc.
%
% please place your own definitions here and don't use \def but
% \newcommand{}{}
%
\journalname{Eur. Phys. J. C}
%
\begin{document}

\title{DarkBit -- a Modular Tool and Interface to Compute Dark Matter Properties%\thanksref{t1}
}
%\subtitle{Do you have a subtitle?\\ If so, write it here}

\titlerunning{DarkBit}        % if too long for running head

\author{Torsten Bringmann, Jan Conrad, Jonathan M.~Cornell, Lars A.~Dal, Joakim Edsj\"o, Miguel Pato, Antje Putze, Christopher Savage, Pat Scott, Christoph Weniger + XXX\thanksref{e1,addr1,addr2}
%        \and
%        Second Author\thanksref{e2,addr2,addr3} %etc.
}

\thankstext{e1}{e-mail: resistanceisfutile@gambit.com}

%\authorrunning{Short form of author list} % if too long for running head

\institute{First address \label{addr1}
           \and
           Second address \label{addr2}
%           \and
%           \emph{Present Address:} if needed\label{addr3}
}

\date{\today}


\newcommand{\DB}{\mbox{\sf DarkBit}}
\newcommand{\GB}{\mbox{\sf GAMBIT}}


\maketitle

\begin{abstract}

Introducing \DB, physics motivation, main features (flexible and efficient interface; more realistic likelihoods
than typically implemented; further new features?). Mention link to \GB\ but emphasize standalone character. 
List examples for illustration...

%\keywords{First keyword \and Second keyword \and More}
% \PACS{PACS code1 \and PACS code2 \and more}
% \subclass{MSC code1 \and MSC code2 \and more}
\end{abstract}

%%%%%%%%%%%%%%%%%%%%%%%%%%%%%%%%%%%%%%%%%%%%%%%%%%%%%%%
%%%%%%%%%%%%%%%%%%%%%%%%%%%%%%%%%%%%%%%%%%%%%%%%%%%%%%%
\section{Introduction [TB/AP/XXX]}
\label{intro}

\begin{itemize}
\item intro to physics: DM and WIMPs; motivation, production + detection methods
\item outline scope of \DB; relations to backends \& genuinely new aspects
\item relation to \GB
\end{itemize}

This article is organized as follows. In Section \ref{phys} we  briefly introduce the general physics 
context and the main  observables to be calculated, before describing in Section \ref{code} the corresponding 
implementation details for the various modules of \DB. Validation tests and illustrative examples of typical
\DB\ usage are presented in Section \ref{examples}. We continue with an outlook on planned code expansions 
with future releases in Section \ref{out}, and conclude in Section \ref{conc}. In \ref{code_init} we provide a quick 
guide for how to install \DB\ and get started with a simple test example.

%%%%%%%%%%%%%%%%%%%%%%%%%%%%%%%%%%%%%%%%%%%%%%%%%%%%%%%
%%%%%%%%%%%%%%%%%%%%%%%%%%%%%%%%%%%%%%%%%%%%%%%%%%%%%%%
\section{Physics framework}
\label{phys}


%%%%%%%%%%%%%%%%%%%%%%%%%%%%%%%%%%%%%%%%%%%%%%%%%%%%%%%
\subsection{Relic density {\bf [JE]}}
\label{phys_rd}

%%%%%%%%%%%%%%%%%%%%%%%%%%%%%%%%%%%%%%%%%%%%%%%%%%%%%%%
\subsection{Direct detection {\bf [CS]}}
\label{phys_dd}

%%%%%%%%%%%%%%%%%%%%%%%%%%%%%%%%%%%%%%%%%%%%%%%%%%%%%%%
\subsection{Indirect detection {\bf [TB]}}
\label{phys_id}

$\to$ general physics \& concepts; specifics go to next two subsections 
(including descriptions of respective likelihood)

%%%%%%%%%%%%%%%%%%%%%%%%%%%%%%%%%%%%%%%%%%%%%%%%%%%%%%%
\subsubsection{Indirect detection with gamma rays {\bf [CW]}}
\label{phys_ga}


%%%%%%%%%%%%%%%%%%%%%%%%%%%%%%%%%%%%%%%%%%%%%%%%%%%%%%%
\subsubsection{Indirect detection with neutrinos {\bf [PS/JE]}}
\label{phys_nu}




%%%%%%%%%%%%%%%%%%%%%%%%%%%%%%%%%%%%%%%%%%%%%%%%%%%%%%%
%%%%%%%%%%%%%%%%%%%%%%%%%%%%%%%%%%%%%%%%%%%%%%%%%%%%%%%
\section{Module description}
\label{code}


%%%%%%%%%%%%%%%%%%%%%%%%%%%%%%%%%%%%%%%%%%%%%%%%%%%%%%%
\subsection{General code framework {\bf [CW]}}
\label{code_gen}

\begin{itemize}
\item connection to \GB, introduce concept of backends, setup and initialization of particle models
\item briefly describe which backends are used, and what they contribute 
\end{itemize}

%%%%%%%%%%%%%%%%%%%%%%%%%%%%%%%%%%%%%%%%%%%%%%%%%%%%%%%
\subsection{Relic density {\bf [JE/TB]}}
\label{code_rd}
\begin{itemize}
\item two possibilities: $W_\mathrm{eff}$ directly from DS or built from process catalogue (as introduced below)
\item...
\end{itemize}

%%%%%%%%%%%%%%%%%%%%%%%%%%%%%%%%%%%%%%%%%%%%%%%%%%%%%%%
\subsection{Direct detection {\bf [CS]}}
\label{code_dd}


%%%%%%%%%%%%%%%%%%%%%%%%%%%%%%%%%%%%%%%%%%%%%%%%%%%%%%%
\subsection{Indirect detection {\bf [CW/TB]}}
\label{code_id}

$\to$ general physics \& concepts; specifics go to next two subsections
\begin{itemize}
\item Describe process catalogue 
\item Different final states: "basic" 2-body vs. cascade {\bf [LD]} vs higher-order corrections
\item...
\end{itemize}

%%%%%%%%%%%%%%%%%%%%%%%%%%%%%%%%%%%%%%%%%%%%%%%%%%%%%%%
\subsubsection{Indirect detection with gamma rays {\bf [CW]}}
\label{code_ga}

%%%%%%%%%%%%%%%%%%%%%%%%%%%%%%%%%%%%%%%%%%%%%%%%%%%%%%%
\subsubsection{Indirect detection with neutrinos {\bf [PS]}}
\label{code_nu}



%%%%%%%%%%%%%%%%%%%%%%%%%%%%%%%%%%%%%%%%%%%%%%%%%%%%%%%
%%%%%%%%%%%%%%%%%%%%%%%%%%%%%%%%%%%%%%%%%%%%%%%%%%%%%%%
\section{Validation and  Examples}
\label{examples}

In this Section we present a few selected  examples that illustrate scope and
potential applications of \DB. At the same time, these examples serve as validation
tests of the code.

%%%%%%%%%%%%%%%%%%%%%%%%%%%%%%%%%%%%%%%%%%%%%%%%%%%%%%%
\subsection{Setting up an effective WIMP model {\bf [CW]}}

$\to$ Briefly describe what we did in Geilo, and maybe reproduce Fig.5 in 1204.3622

\medskip
...

%%%%%%%%%%%%%%%%%%%%%%%%%%%%%%%%%%%%%%%%%%%%%%%%%%%%%%%
\subsection{Comparing {\sf DarkSUSY} and {\sf micrOMEGAs} {\bf [JE/JC]}}

\DB\ offers the unique possibility to easily compare different numerical codes for the computation of DM properties
in a well-defined and consistent way. Here, we focus for illustration on the most widely used packages, {\sf DarkSUSY} 
\cite{xxx} and {\sf micrOMEGAs} \cite{xxx}. We stress, however, that it is straightforward for users to 
perform similar comparisons for essentially any other numerical code, simply by adding it as a backend to \DB.

\medskip
$\to$ focus on relic density, DD couplings, maybe spectra

\medskip
...


%%%%%%%%%%%%%%%%%%%%%%%%%%%%%%%%%%%%%%%%%%%%%%%%%%%%%%%
\subsection{Gamma-ray spectra from cascade decays {\bf [LD/MP]}}

One of the main new features of this first \DB\ release is a novel way of calculating the cosmic-ray spectra that result from 
cascade decays of the primary products of DM annihilation or decay. For illustration we focus here on gamma
rays as final states, and compare in some detail the spectra obtained with \DB\ with those from other codes
as well as analytical expectations.

\medskip
...


%%%%%%%%%%%%%%%%%%%%%%%%%%%%%%%%%%%%%%%%%%%%%%%%%%%%%%%
\subsection{Benchmark points for DM models {\bf [all]}}

\subsubsection{MSSM}
...

\subsubsection{Scalar Singlet DM}
...

%%%%%%%%%%%%%%%%%%%%%%%%%%%%%%%%%%%%%%%%%%%%%%%%%%%%%%%
\subsection{more examples ?}
...

%%%%%%%%%%%%%%%%%%%%%%%%%%%%%%%%%%%%%%%%%%%%%%%%%%%%%%%
%%%%%%%%%%%%%%%%%%%%%%%%%%%%%%%%%%%%%%%%%%%%%%%%%%%%%%%
\section{Outlook {\bf [all]}}
\label{out}

...

%%%%%%%%%%%%%%%%%%%%%%%%%%%%%%%%%%%%%%%%%%%%%%%%%%%%%%%
%%%%%%%%%%%%%%%%%%%%%%%%%%%%%%%%%%%%%%%%%%%%%%%%%%%%%%%
\section{Conclusions {\bf [all]} }
\label{conc}

...

\begin{acknowledgements}
...
\end{acknowledgements}



%%%%%%%%%%%%%%%%%%%%%%%%%%%%%%%%%%%%%%%%%%%%%%%%%%%%%%%
%%%%%%%%%%%%%%%%%%%%%%%%%%%%%%%%%%%%%%%%%%%%%%%%%%%%%%%
\appendix
\label{}

%%%%%%%%%%%%%%%%%%%%%%%%%%%%%%%%%%%%%%%%%%%%%%%%%%%%%%%
\section{Getting started  {\bf [AP/XXX]}}
\label{code_init}

$\to$ Installation, how to run test example

\medskip
...


% BibTeX users please use one of
%\bibliographystyle{spbasic}      % basic style, author-year citations
%\bibliographystyle{spmpsci}      % mathematics and physical sciences
%\bibliographystyle{spphys}       % APS-like style for physics
%\bibliography{}   % name your BibTeX data base

% Non-BibTeX users please use
\begin{thebibliography}{}
%
% and use \bibitem to create references. Consult the Instructions
% for authors for reference list style.
%
\bibitem{RefJ}
% Format for Journal Reference
Author, Article title, Journal, Volume, page numbers (year)
% Format for books
\bibitem{RefB}
Author, Book title, page numbers. Publisher, place (year)
% etc
\end{thebibliography}

\end{document}
% end of file template.tex

