\documentclass[11pt,a4paper]{article}

\usepackage{fancyhdr}
\usepackage{parskip}
\usepackage[usenames,dvipsnames]{color}
\usepackage[colorlinks=true, linkcolor=BrickRed, citecolor=Blue, urlcolor=Blue, filecolor=Blue]{hyperref}
\usepackage{graphicx}
\usepackage{xcolor}
\usepackage{amssymb, amsmath}
\pagestyle{fancy}

\newcommand{\mail}[1]{\href{mailto:#1}{#1}}

\begin{document}

\thispagestyle{fancy}

%Include the logo in this code when it is finished
\centerline{\textbf{GAMBIT: Global And Modular BSM Inference Tool}}



\centerline{\textbf{Policy and Procedures Document}}\bigskip 


\section{Membership}

\subsection{Collaboration Size}

Members will keep in mind that the Collaboration should not grow to a size that makes it difficult to manage, either scientifically or administratively.  This present definition of this bound is a maximum size of 30 members.

\subsection{Adding members}

New members will be proposed to the full Collaboration officially, via email to the Collaboration mailing list.  Members will respond either positively or negatively by email, or abstain.  If consensus in favour of accepting the nomination is reached by email, as voted by at least 50\% of existing members after a period of 1 week since the nomination email, then the nomination is accepted.  If not, a Vidyo meeting will be called to discuss the case, with the goal of reaching consensus.  If consensus has not been reached in an hour, any member may call for a vote.  If 2 or more members vote against accepting the suggestion for new membership, then the suggestion is immediately rejected.  Otherwise, if at least 50\% of members vote in favour of accepting the suggestion rather than abstaining, it is conditionally accepted.  Under normal circumstances, votes from members unable to attend the meeting will be accepted for \textbf{1 week following the meeting}.  If 2 or more existing members have officially voted against accepting the membership suggestion by the end of the second voting period, or less than 50\% of members have officially voted in favour of accepting, then the suggestion is rejected; otherwise, it is accepted in full.  If a member is absent from email during either the first or second voting period, either as advised to the Collaboration mailing list or as informally advised by another member, the relevant period will be extended until 3 days after their return, or whenever they vote (whichever is sooner).

Suggestions for new members should be motivated by the following criteria:
\begin{enumerate}
\item the new member should bring specific missing physics expertise to the Collaboration, and/or
\item the new member will pull their weight in terms of coding and code design
\end{enumerate}
Whilst both criteria need not necessarily be met in every case, there is a strong preference for prospective members who do meet both.  

The workflow for a adding a new member is as follows:
\begin{enumerate}
\item Email the full Collaboration mailing list with the suggestion, addressing the criteria above
\item Argue the case at the subsequent Vidyo meeting (if required)
\item Tell a Wiki Manager to add the new member as an editor of the Wiki
\item Tell a Repository Manager to add the new member to the code repository and mailing lists
\item Add the new member's details to the document \textit{Members} in the Collaboration Docs section of the repository
\end{enumerate}

This policy applies equally for all suggested new members, regardless of academic rank.  In particular, it applies even if the suggested members are current students of existing members.  Membership for students of existing members, whilst generally encouraged if the student meets the criteria, will not necessarily be any more `automatic' than for any other researcher.

\subsection{Removing members}

Membership of GAMBIT can be renounced voluntarily at any point, simply by emailing the whole Collaboration and requesting to be removed from the list of members.  This may be appropriate if a member e.g.\ leaves physics, takes a job that leaves them no time to contribute to the Collaboration, or simply decides to pursue alternative research interests.

Members can also be forcibly removed from the Collaboration under the following circumstances:\begin{itemize}
\item Re-release of the GAMBIT code under an alternative name or otherwise against Collaboration wishes 
\item Publication, presentation or other dissemination of Collaboration work without previously informing the Collaboration, seeking its approval, or appropriately crediting the Collaboration's role in the work -- or, grievously misrepresenting the views of the Collaboration in said dissemination
\item Significantly damaging the Collaboration's competitive position with regard to other groups 
\item Becoming wholly inactive or otherwise effectively impossible to work with
\end{itemize}
WG Coordinators are expected to watch for these and report them to the rest of the Collaboration.

The Collaboration will operate a `2 strikes' policy in general, with a warning typically given for a first infringement, and ejection to result from the second.  Extreme cases can however lead to immediate expulsion.

In the case of a claimed breach of the code of conduct implied by the above list, the Collaboration will call a meeting over Vidyo to discuss the case, and eventually vote on whether or not to warn/expel the offending member.  Issuing a warning or removal will require a majority decision amongst those who vote.  Members unable to attend the meeting will be given \textbf{2 weeks to submit their vote by email}.

\subsection{Percentage time commitment}

Members are required to give an indication of the percentage of their working time that they intend to devote to Collaboration work when they sign on, and continue to update this commitment as that percentage changes.  This requirement exists so that members know what to expect from their collaborators, and so that each member is forced to continuously assess whether or not they are presently fulfilling their stated commitment to their collaborators.

Absences (from both email and direct contact) are \textit{requested} to be reported to the full Collaboration in advance via email for all periods of \textbf{a week or more}.  Absences of \textbf{more than a month} \textit{must} be indicated in advance to the whole Collaboration by email.

\subsection{Classes of membership}

Membership does not consist of subclasses for students, affiliates, etc; researchers will either be members or non-members.

Membership can however be designated either `active' or `adjunct'.  The adjunct category is intended as a special accommodation for members who either
\begin{enumerate}
\item foresee a temporary period of $\gtrsim$4 months where their time commitment to GAMBIT will be approximately 0\% (e.g. extended absence, caring responsibilities, etc), followed by an anticipated switch back to active status, or
\item cannot reasonably commit to any active engagement in GAMBIT activities, be that coding, scientific work or meeting participation, but who still wish to retain an association with the Collaboration, and with whom the Collaboration sees a defined benefit in maintaining an ongoing association.
\end{enumerate}
It is expected that adjunct members will abide by all the same Collaboration rules and guidelines as active members, except that they will not generally be expected to advise of periods of non-contactability.  In particular, they will be expected to continue to act in the best interests of the Collaboration, and help promote its work wherever reasonably possible.

Adjunct members will not receive automatic invitations to sign Collaboration papers (neither long nor short author), and will only be invited to sign those papers where they have made a direct and significant contribution.  Adjunct members will not generally be expected to contribute to the day-to-day activities of the Collaboration in terms of meeting attendance, code or science development, although they are certainly encouraged to do so whenever they feel able.  They will have full access to all Collaboration meetings and resources, including Wiki and repository access.

Members can switch between active and adjunct status at essentially any time, but the switch should not be made lightly or regularly; adjunct status should typically be used sparingly, and only when truly required.  Adjunct status should be used as an alternative to resignation/ejection from the Collaboration in cases where the member still has something to offer but cannot fulfil the responsibilities of an active member.  It should not be used simply to `soften' an ejection, or as a disciplinary measure.

Plans to switch between active and adjunct status (in either direction) should be notified to the Collaboration-wide email list, and then need to be ratified by a vote at the next all-collaboration Vidyo meeting.  Approved status changes will take place immediately, and unlike other votes, there will be no 2-week grace period for members absent from the meeting to place a late vote by email (although they may do so in advance of the meeting, by email).

Adjunct members will not count against the membership cap.

It will not generally be possible to join a new member up directly as `adjunct'; new members should be accepted only when it can be reasonably expected that they will begin to make an active contribution immediately after joining.

\subsection{Working Groups}

The Collaboration contains multiple internal Working Groups (WGs), centred on different code responsibilities (specific physics modules, scanning, core, etc).  Members should sign themselves up to one or more WGs, and are free to move between WGs as they like over the course of their membership in the Collaboration.

WGs will have one or two coordinating leaders (WG Coordinators).  Coordinators will have responsibility for:
\begin{itemize}
  \item setting coding (and later, analysis) goals and priorities for the WG
  \item dividing tasks amongst WG members
  \item tracking WG coding (and later, analysis) progress
  \item tracking contributions of individual members to WG tasks, with particular view to\begin{itemize}
  \item ensuring that all \textit{non-}WG members who actually have contributed to a given WG-centric short author paper are offered authorship of that paper
  \item asking WG members who have not really contributed to such papers to consider not signing them
  \item identifying inactive members 
  \item prompting inactive members to become active again
  \item suggesting to the Collaboration that inactive members be asked to leave, or ejected 
  \end{itemize}
\end{itemize}

Although no specific tenure will be set for WG Coordinators, they should be discussed at each Collaboration Meeting, with a view to eventually rotating members through the Coordinator roles.

\section{Publication policy}

\subsection{Code Release}

As a general rule, code will be released in full simultaneously with the first paper to utilise it.  In the case of individual physics modules, it is intended that each module will have a corresponding code paper written about it, and the paper and module will be released together.  This will happen before the full GAMBIT code, accompanying code paper and accompanying first physics paper are released, except in the case of the HEColliderBit module.  This module and accompanying code paper will be released simultaneously with the full GAMBIT code, code paper and first physics analysis.

The following publication rules for papers and presentations apply equally before and after the full GAMBIT code has been released.

\subsection{Papers}

\subsubsection{Categories -- Long and Short}

Papers using parts of the GAMBIT code (except for single modules), or lists of samples produced with it, must be proposed to the whole Collaboration by email before any significant work is done on the project, as either long or short author papers.  

Long author papers will contain the entire Collaboration author list, arranged alphabetically.  They may also contain external authors (e.g. students who are not Collaboration members themselves, but happen to be supervised by a Collaboration member, or external collaborators), pending Collaboration approval.  Long author papers will typically involve the first global fit of a particular model, a significant new result, or inclusion of a significant new observable or dataset.

Short author papers will contain the subset of Collaboration members who have directly contributed to the analysis, arranged however the authors prefer.  Short author papers may also contain additional external authors, which may be added and removed without subsequent Collaboration approval.  Short author papers will typically be more specific or directed analyses, involving no new code, only a small addition to the code, or a less significant spin-off analysis from one of the long-author papers.

There is no default publication mode; each case must be discussed on its individual merits and the category decided within the Collaboration.  Within their body of work involving GAMBIT, members are however expected to clearly prioritise long author papers, as the key results to come out of the code are expected to be of this category.

If Collaboration members do not agree by email on the categorisation of a proposed paper as long or short, then a Vidyo meeting will be called to vote on the paper category.  In this case, the category with the greater number of votes will be adopted.  In the case of a tie, the paper will be categorised as long author.  If a paper is categorised as long, a member can still choose to extract themself from the project and resulting author list, making it a short author paper, but with all remaining members of the Collaboration as authors.

Module code papers will be published as short author papers.  Single modules will be permitted to be used for future analyses without Collaboration permission, or proposal of the corresponding paper to the Collaboration as a short author paper.  The applies especially, but is not limited, to the original authors of a module.\footnote{A module is defined as one of the code packages referred to as such in the document \textit{Working Groups and Individual Positions}.}  Using more than one module in a single paper requires proposing the paper to the Collaboration as described above.  This requirement is not designed to apply to non-GAMBIT codebases used as backends for multiple modules, only to the module codes themselves.  A non-likelihood analysis using e.g.\ parallel Herwig to calculate better-sampled yield tables for DarkSUSY, and a subsequent comparison of indirect dark matter search limits to direct limits, would not need to be proposed to the Collaboration unless more than one of the GAMBIT modules themselves were used for communication between the different backends.  Even if multiple GAMBIT modules were used in such a study, and it therefore required proposing to the Collaboration, this project would be an obvious example of one appropriate for publication as a short author paper.

Users of modules will be asked to cite both the relevant module code paper, and main (long author) code paper.

\subsubsection{Review}

Long author papers will be officially reviewed internally within the Collaboration before posting to the arXiv or submission to a journal.  One internal reviewer will be selected for this task, for each paper.  The reviewer will have \textbf{2 weeks} to carry out the review, and must officially approve a version of the paper before submission to the arXiv or a journal.  The paper will also be sent out to the whole Collaboration for comment during \textbf{the same 2 week period}.

The Publication Manager will be responsible for selecting internal referees, and giving the final approval or otherwise for posting/submission following the review process and comment period.  If serious concerns arise either in the review or comment process, they are expected to be resolved before a paper is approved for posting/submission.

Short author papers do not require internal review, although it may be requested.  Circulation of the short author paper within the Collaboration for comment will only be required in advance of posting to the arXiv or submission to a journal, if it was explicitly decided at the time the project was originally classified as short author, that this would be desirable.  In this case the period for circulation will be \textbf{one week}.  Authors of short author papers are nonetheless encouraged to seek the full Collaboration's comments on drafts before submission in any case.

\subsubsection{Submission}

Papers will usually be submitted simultaneously to the arXiv and a journal.

\subsubsection{Conduct}

At the very least, all authors of any GAMBIT paper (long or short author) are \textbf{required} to read the paper in full and give feedback to the lead authors.  Members unable to do this will be asked to remove themselves from the author list.

In general, gaming of the Collaboration paper publication policy is understood to result in a severe ass kicking.

\subsection{Presentations}

Talks to be given on behalf of the Collaboration are to be circulated for comment (to the full Collaboration) a minimum of \textbf{72\,hr in advance} of the talk.  It is expected that talks circulated in this way will be very close to complete.  All the potentially controversial bits \textbf{must} be sufficiently fleshed out to allow useful comment.  Members are expected to take on board criticisms and suggestions for modifications from other members.  

The final form of all talks given on behalf of the Collaboration should be uploaded to the Wiki.

The Publication Manager will be responsible for ensuring that talks are circulated sufficiently far in advance, and appropriately archived.  He/she can also enforce the removal or modification of content in talks if truly necessary.

If the talk is not officially on behalf of the Collaboration, but touches on Collaboration work, sending it out would be a nice courtesy to the rest of the Collaboration, but is not compulsory.  The Publication Manager has no power to compel the speaker with regard to content in this case.

The Publication Manager will also keep track of upcoming conferences, identify those where a GAMBIT contribution would be desirable, and encourage relevant members to submit an abstract.  The Publication Manager will likewise serve as a central administrator of invitations to present; members who have been invited to give a presentation on GAMBIT but cannot, and cannot identify a suitable replacement, should pass the invitation on to the Publication Manager, who will attempt to find a member who can accept the invitation.


\section{Meetings}

Full Collaboration Meetings will be held approximately every 9 months.  Although no specific tenure is given for WG Coordinators, all positions mentioned in the document \textit{Working Groups and Individual Positions} (i.e.\ WG Coordinators and non-WG individual positions) will be discussed at each Collaboration Meeting, with the goal of establishing an orderly rotation of roles.

All-Collaboration Vidyo meetings will be held every 4 weeks.

WGs should meet/discuss amongst themselves on a shorter timescale.  


\section{Wiki}

The GAMBIT Wiki should contain 
\begin{itemize}
\item lists of current papers/projects (both short and long author, approved and proposed)
\item past talks given on behalf of the Collaboration (.ppt or .tex+.pdf+associated files)
\item resources for preparing grant applications relating to GAMBIT (including copies of past grant applications)
\item code progress
\item bug tracking
\item member list
\item recorded outcomes of official votes (new member decisions, etc)
\item information about / links to coding resources (OpenCL, OpenMP, etc)
\end{itemize}

\end{document}
