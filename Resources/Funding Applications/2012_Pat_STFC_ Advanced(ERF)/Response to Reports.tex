\documentclass[11pt,oneside,twocolumn,a4paper]{article}

\usepackage{graphicx}
\usepackage[footnotesize,bf]{caption}
\usepackage{fancyhdr}
\usepackage[numbers,sort&compress]{natbib}
\usepackage[usenames,dvipsnames]{color}
\usepackage{enumitem}
\usepackage[colorlinks=true, linkcolor=BrickRed, citecolor=Blue, urlcolor=Blue, filecolor=Blue]{hyperref}
\input{jdefs.tex}

\addtolength{\topmargin}{-0.5in}
\addtolength{\textheight}{1.0in}
\addtolength{\textwidth}{0.5in}
\addtolength{\oddsidemargin}{-0.5cm}

\fancyhead[LE,RO]{\thepage}
\fancyhead[LO,RE]{Pat Scott}
\fancyhead[C]{STFC ERF Application: PI Response to reviewer reports}
\fancyfoot[C]{}
\renewcommand{\headrulewidth}{0.4pt}
\renewcommand{\footrulewidth}{0pt}
\pagestyle{fancy}

\author{Pat Scott}
\date{}

\setlist{nolistsep}

\begin{document}

\thispagestyle{fancy}

I thank the reviewers for their strong support of my proposal.  \vspace{4mm}

I am pleased by the support offered in Report \textbf{1-14IBSH}.  As no queries were raised here, I have no further comments on this review.\vspace{4mm}

Report \textbf{1-15B2SU} asks about a backup plan if evidence of BSM physics is not discovered in the term of my ERF.  This was mentioned briefly in the Case for Support (see the final sentence on p1), but I can provide further details here.  My proposal does not depend on new physics actually being discovered over the next few years.  Even without a detection, I can and will combine the various non-detections in the rigorous global fits I describe, then use these fits to make robust and concrete statements about which BSM models and parameter subspaces are excluded, and which remain viable.  This will not only inform future theoretical model-building, but also point the way for experimentalists developing novel techniques for probing future and remaining BSM models.  \vspace{4mm}

Reviewer \textbf{1-14IBSI} is concerned about whether there will be enough involvement of experimentalists in my research program to achieve the level of detail and sophistication I claim in the likelihood modules.  This is a fair question, and one that I have actually been busy addressing in the past few months.  In this period I have formed and now lead a formal collaboration designed to facilitate the development of these likelihood modules.  The collaboration includes full or affiliate members of ATLAS (A.~Buckley, C.~Clement, P.~Jackson, A.~Saavedra, M.~White), CMS (C.~Rogan), HESS (J.~Conrad, H.~Dickinson), \textit{Fermi}-LAT (myself, J.~Conrad, J.~Edsj\"o, G.~Martinez), IceCube (myself, J.~Edsj\"o, C.~Savage), AMS-02 (A.~Putze) and CTA (T.~Bringmann, J.~Conrad, H.~Dickinson), as well as a number of theorists/phenomenologists (myself, C.~Bal\'azs, T.~Bringmann, J.~Edsj\"o, N.~Mahmoudi, A.~Raklev, C.~Weniger and our respective students and postdocs).  We are in the process of also recruiting a member of LHCb.  As an ERF at Imperial, I would also expect to bring one or more representatives of LUX/LZ into this collaboration, along with additional CMS members.    Combining this collaboration with the additional experimental expertise at Imperial will provide extremely close links to a very broad range of BSM experiments.

The reviewer specifically asks of the likelihood modules: 

\vspace{3mm}\noindent\textit{``Will these have to be written by collaborators on the relevant projects who have detailed knowledge of the data processing and systematic effects?''}\vspace{3mm}

Many of the modules will be written \textit{together with} collaborators in the respective experiments, drawing on the unique insights and experience they have to offer (on topics such as data processing and systematics), but not exclusively \textit{by} them.

\end{document}
