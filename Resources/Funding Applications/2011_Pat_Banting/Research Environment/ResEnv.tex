\documentclass[10pt,oneside,onecolumn,a4paper]{article}

\usepackage{graphicx}
\usepackage{fancyhdr}

\addtolength{\topmargin}{-.5in}
\addtolength{\textheight}{1.2in}

\fancyhead[LE,RO]{\thepage}
\fancyhead[LO,RE]{\slshape Pat Scott}
\fancyhead[C]{Banting PDF -- Research Environment}
\fancyfoot[C]{}
\renewcommand{\headrulewidth}{0.4pt}
\renewcommand{\footrulewidth}{0pt}
\pagestyle{fancy}

\pagestyle{fancy}

\begin{document}

\bigskip

\subsubsection*{McGill University}

McGill University is an internationally recognized, world-class
research institute. It is currently ranked the top university in
Canada and is among the top twenty universities in the world (according
to the QS World University Rankings). The undergraduate population
is exceptional, as manifested for example by the fact that several
former McGill undergraduates were recently awarded Nobel Prizes. There
are extensive funding resources for graduate students and postdoctoral
fellows, fostering an active and stimulating research and
academic environment for Dr.\ Scott.

\subsubsection*{The Department of Physics}

McGill's Department of Physics is a well-established and world-renowned
department within McGill. It is one of the oldest physics departments
in Canada, and has a record of forefront research going back to
Rutherford.  It has produced numerous Nobel Laureates, and shown
consistent commitment to research excellence.

In the past decade the McGill Physics Department has undergone an
extensive faculty renewal. Currently, more than 60\% of the Physics
faculty are people who were hired since the year 2000. Without
exception, these faculty members are very active in research. The research
funding level per faculty is the highest of any Canadian Physics department.
This has enabled the Physics Department to recruit a large number
of graduate students and postdocs. There are close to 150 graduate
students, and thus McGill has the highest ratio of graduate students
to faculty among Canadian Physics departments. The number of physics
majors is exceptionally high (more than 60 per year) and of excellent
quality (as measured e.g. by the fraction who continue to graduate
school and by how well they then perform in graduate school). All
these factors provide Dr.\ Scott with an extremely stimulating research
environment.

\subsubsection*{Personnel}

In the year 2000, McGill decided to invest in a new astrophysics
group and has provided all of the resources required for this group
to grow and form a true centre of excellence. It currently consists
of eight faculty members (six of whom were hired in the period
between 2000 and 2006), eight postdocs and more than 20 graduate
students. At least four of the faculty have close research overlap
with Dr.\ Scott: two of the faculty (Hanna and Ragan) are playing a key role in 
the high energy gamma ray telescope VERITAS collaboration, Professor
Dobbs is one of the leaders in several cosmic microwave (CMB) temperature
and polarization experiments, and Professor Holder is an expert
on theoretical aspects of CMB and cosmic reionization research. 

In the years following 2000, McGill also hired three new faculty in
experimental particle physics (currently playing an important
role in the ATLAS collaboration at the Large Hadron Collider). In the same period,
the High Energy Theory group underwent a renewal with the hiring of
Profs.\ Brandenberger, Moore, Dasgupta, Maloney and Walcher. Three
of these High Energy Theory faculty have close research overlaps
with Dr.\ Scott.  Foremost is Prof.\ Cline, who is currently working
on dark matter models as his main research priority.  Prof.\ Brandenberger
is a world-renowned expert on both early Universe and late time cosmology, both
of which constitute essential inputs to Dr.\ Scott's proposed program of global fits. Prof.\ Moore is an expert on
quantum field theory at finite temperature and density, which is also highly relevant for physics of the early Universe.

One of the most special things about the current research environment in the
McGill Physics Department is the close connection between all research
groups with interests relevant to Dr.\ Scott's
proposal. The High Energy Theory group interacts very closely
with the Astrophysics Group, in particular concerning topics related
to cosmology. Profs.\ Cline and Brandenberger are both directly
collaborating with members of the Astrophysics Group. Members of
the High Energy Theory group also maintain an active research collaboration
with the Nuclear Theory group, led by Profs.\ Gale and Jeon.  This particular resource will be especially valuable to Dr.\ Scott, because nuclear form factors are presently one of the largest uncertainties in the search for neutrinos from dark matter annihilation in the Sun, and in direct detection experiments.  Thus, McGill provides Dr.\ Scott with a unique research environment, where all
groups working on issues relevant to his research continuously collaborate
and exchange ideas.

At McGill, Dr.\ Scott finds an ideal environment for the
research program he plans to carry out. His research must draw
on the expertise of people in quite varied areas: dark matter theory
(Prof.\ Cline), indirect detection (Profs.\ Hanna and Ragan),
cosmological aspects (Profs.\ Brandenberger and Holder from the
theory side and Prof.\ Dobbs from the experimental side), 
collider constraints (Profs.\ Vachon, Warburton, Corriveau and
Robertson from the ATLAS group), and quantum field theory
(Professors Cline, Moore, Jeon and Gale). Experts on direct
dark matter detection are located across Mt. Royal at
the Universit\'e de Montr\'eal (the PICASSO experiment). This unique
combination of in house expertise and research strengths
is extremely valuable for Dr.\ Scott (who turned down a postdoctoral
fellowship at Harvard to come to McGill, specifically because he felt
McGill offered a superior environment for his research program).

Likewise, McGill also has a special interest in having
Dr.\ Scott here in Montreal. Dr.\ Scott is \textbf{THE} person who can provide
the link between all of the research groups mentioned above. In
particular, he is forging new links (e.g. a new seminar series)
between particle physics, astrophysics and high energy theory.

At McGill Dr.\ Scott has the chance to work with many excellent
graduate students, both from the High Energy Theory and the Astrophysics 
groups. He has already identified a core group of 4 students that would be heavily
involved in the work he proposes here.

\subsubsection*{Facilities and Funding}

Dr.\ Scott will have full access to a computer cluster recently
purchased by the High Energy Theory group with applications to
astrophysical and cosmological simulations in mind. The Astrophysics
Group has a larger cluster used extensively for data analysis, which Dr.\ Scott will also have full access to. 
The Department provides two computer technicians who look after these computer
clusters. Thus, there are excellent computational resources within
the Department for Dr.\ Scott's research. McGill is a member of the
CLUMEQ High Performance Computing network (linking several Quebec
institutions), and the new director of CLUMEQ is a member of the
Physics Department. Dr.\ Scott will also have access to all of CLUMEQ's
facilities.

Dr.\ Scott will be provided with a \$10k annual research budget, which can be used for travel, computing hardware or inviting collaborators.  The Department has 5 full-time administrative staff, all of whom will be available to assist Dr.\ Scott in administering his Fellowship, whether regarding paperwork, travel, workshop/conference organization or hosting guests.

\end{document}
