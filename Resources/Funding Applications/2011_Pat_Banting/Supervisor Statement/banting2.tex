%\documentstyle[12pt]{article}
\documentclass[12pt]{article}
\usepackage{graphicx}
\usepackage{latexsym}
\pagestyle{myheadings}
\markright{{Pat Scott, Banting PDF Supervisor's statement -- James M. Cline}}
\textheight 23.7cm
\textwidth 17.7cm
\headsep 10mm
\oddsidemargin -0.25in   % margins on all sides of 3.5 cm,
\evensidemargin -0.25in  % correcting 1 inch default on left-hand
\topmargin -0.25in   % side, and 1.5 inch default on top.
\let\oldbibliography\thebibliography
\renewcommand{\thebibliography}[1]{%
  \oldbibliography{#1}%
  \setlength{\itemsep}{0pt}%
}

%\setcounter{page}{8}

\begin{document}

\parindent 0cm
\parskip 10pt
\baselineskip 12pt

\subsubsection*{Supervision of the Proposal}

A supervisor with versatile background and skills is needed to bring
Dr.\ Scott's ambitious research proposal to fruition.  My
combination of expertise in theoretical cosmology and particle
physics is the strongest reason Dr.\ Scott chose to come to McGill instead of
four other competing institutions (including the offer of a named
fellowship at the Harvard Center for Astrophysics).  Dr.\ Scott's work
is very much at the interface of these two fields.  Although there
are many potential supervisors who could lead on the
astrophysics/cosmology front, the guidance of an eminent particle
phenomenologist is also essential for the seminal aspect of his proposal:
comparing predictions of broad classes of particle physics models for 
the identity of dark matter (DM).

My current research interests strongly overlap those of Dr.\ Scott,
and I can provide essential guidance to him in the course of his
research program.  We have been intensively discussing the
intricacies of DM model building during the planning stages
of our collaboration, which we are now beginning in earnest.
Much of the proposal relates to DM candidates from the
Minimal Supersymmetric Standard Model (MSSM), which was the framework
where I made formative contributions to the theory of electroweak
baryogenesis \cite{Cline:1996cr}-\cite{Cline:2000nw} (656 citations). 
This work was carried out partly at CERN and tested to a great extent
by the CERN LEP and Tevatron experiments, through their searches for
the Higgs boson, light neutralinos, and top squarks.  It kept my finger on
the pulse of these experimental developments, which continue to unfold
at CERN's Large Hadron Collider (LHC), and will undoubtedly play
a key role in the global fits program that Dr.\ Scott is proposing. 
Part of his present proposal is to explore models that can
accommodate both electroweak baryogenesis and DM, so my
expertise will be indispensable in this aspect.   Moreover, I have
long-standing connections to the CERN theory group, which I am
currently visiting, and where I have a been a Scientific Associate
in 2002-2003 and 2010; these will facilitate Dr.\ Scott's accessibility
to latest results from the LHC as they come (for example via the
Collider Crosstalk forum), well in advance of the
official publications.  Further leadership on the particle theory side was
demonstrated by my election to the council of the 
Institute for Particle Physics (Canada) for a three year term in 2006.

Since 2009, inspired by exciting experimental indications of indirect
DM detection in our Galaxy, my research has focused primarily
on the study of hidden sector DM models for explaining cosmic
ray anomalies and hints of direct DM detection
\cite{Chen:2009dm}-\cite{Cline:2011uu}.  DM in this class of models
has quite different properties from the MSSM DM candidates previously
studied by Dr.\ Scott, and my experience here will be vital for
extending the global fits programs to include these kinds of models,
especially ones in which the DM has isospin-violating interactions
with nuclei \cite{Cline:2011zr,Cline:2011uu}.  I am also an expert on
the possible effects of substructure in the DM halo 
\cite{Cline:2010ag,Vincent:2010kv}, which can be useful for
understanding the constraining power of \emph{Fermi}-LAT gamma-ray
observations on these models, as well as relevant issues concerning
the behaviour of the DM halo in the Galactic centre
\cite{Chen:2009av,Cirelli:2010nh}. A new collaboration with P.\
Martin of MPE, Munich will allow us to access 511\,keV observations of
the INTEGRAL/SPI experiment directly, and thus use this more
effectively as one of the constraints we can impose on  DM models in
Dr.\ Scott's global analysis program.

My proficiency in theoretical cosmology has always been directed toward
phenomenological issues, of relevance to Dr.\ Scott's proposal, such
as understanding possible features in the cosmic microwave background
(CMB) \cite{Cline:2006db}-\cite{Burgess:2002ub}, and comparison of
predictions of theoretical models of inflation to current data
\cite{BlancoPillado:2006he,BlancoPillado:2004ns}. The CMB provides
crucial constraints on DM models, especially light DM, which can change the reionization history of the early
Universe through its annihilation.  These constraints figure prominently in Dr.\ Scott's proposal. I have
been a principal organizer of cosmology conferences of international 
stature at the Aspen Center for Physics (2002) and the Banff
International Research Station (2004).  I was Principal Investigator
for the FQRNT team grant on the Dark Energy of the Universe 
(\$206k, 2005-2007).   I was an invited speaker at the 
international conferences COSMO `08 (Madison) and The Dark Universe
(Heidelberg, 2011).  My accomplishments in these
areas were recognized by the early award of full professor status
in 2006, and a visiting scientist position with the cosmology and particle theory
groups at Perimeter Institute during the latter half of 2010.

\subsubsection*{Departmental Synergy}

Our department provides a remarkable synergy of research efforts that
meaningfully overlap with Dr.\ Scott's research proposal. Within our
group, we have the
benefit of Robert Brandenberger's internationally recognized
leadership in fundamental aspects of theoretical cosmology.  In theoretical astrophysics, Gil Holder (with whom Dr.\ Scott
and I have both collaborated) is extremely active in CMB physics,
whose relevance to Dr.\ Scott's interests I already emphasized.
Likewise in experimental astrophysics, Matt Dobbs is a leading figure
on new experiments (the South Pole Telescope and POLARBEAR) to measure
CMB polarization.  In experimental high-energy astrophysics, David
Hanna's and Ken Ragan's work on the VERITAS experiment provides
another constraint on the properties of DM by its possible
annihilation in nearby astrophysical objects, such as dwarf galaxies.  Of course the postdocs and
students working in these groups greatly multiply the opportunities for
fruitful interactions.

\subsubsection*{Professional, Research and Career Support}

Dr.\ Scott is already performing at an extremely high level for a
postdoc at such an early stage of his career, and the extra research
skills and connections that he can acquire with my assistance will
help to assure his evolution toward a position of leadership at the
highest levels.  He is developing his grant-writing skills by giving
me feedback on my own proposal for a renewal.  He will more easily
establish relationships at CERN and Perimeter Institute via my already
existing links to those institutions.  I can help him to obtain access
to significant computational resources with CLUMEQ, a local
research consortium for high performance computing.  I will be able to
mentor him during the process of interviewing for permanent positions,
which could very well begin before the end of his tenure at McGill.
I can also help him to hone his teaching skills, which will give him an edge
in the keen competition for tenure-track jobs.

A very direct advantage that I am able to offer to Dr.\ Scott stems
from my having a relatively  large number of graduate students
(six), some of whom are already well-versed in DM physics,
others of whom are new to the field and eager to start learning and
contributing.  Ph.D.\ student A.\ Vincent has already been
collaborating with him on a project about DM effects in the
Sun, and M.Sc.\ student Grace Dupuis is starting work with Dr.\ Scott on
DM annihilation in the Galactic centre.  I will try to
recruit one or two new students in the next year to compound these
efforts.  In addition to
providing this manpower toward the realization of his ideas,
we offer generous funding toward research expenses of \$10k per year,
including the expenses of visits by external collaborators.  In short, we
are quite fortunate to have Dr.\ Scott in our department, and I will continue to do all
that is possible to further his development and research goals.


\rightline{\includegraphics[width=0.2\textwidth]{sig.eps}}
\rightline{James M. Cline}
\rightline{Professor of Physics}

\small 

\begin{thebibliography}{99}

%\cite{Cline:1996cr}
\bibitem{Cline:1996cr}
  J.~M.~Cline, K.~Kainulainen,
  ``Supersymmetric electroweak phase transition: Beyond perturbation theory,''
  Nucl.\ Phys.\  {\bf B482}, 73-91 (1996).
  [hep-ph/9605235].  (124 citations)

%\cite{Cline:1997bm}
\bibitem{Cline:1997bm}
  J.~M.~Cline, K.~Kainulainen,
  ``Supersymmetric electroweak phase transition: Dimensional reduction versus effective potential,''
  Nucl.\ Phys.\  {\bf B510}, 88-102 (1998).
  [hep-ph/9705201].\\ (47 citations)


%\cite{Cline:1997vk}
\bibitem{Cline:1997vk}
  J.~M.~Cline, M.~Joyce, K.~Kainulainen,
  ``Supersymmetric electroweak baryogenesis in the WKB approximation,''
  Phys.\ Lett.\  {\bf B417}, 79-86 (1998).
  [hep-ph/9708393] (130 citations)

\bibitem{Cline:1998hy}
  J.~M.~Cline, G.~D.~Moore,
  ``Supersymmetric electroweak phase transition: Baryogenesis versus experimental constraints,''
  Phys.\ Rev.\ Lett.\  {\bf 81}, 3315-3318 (1998).
  [hep-ph/9806354].\\ (115 citations)

\bibitem{Cline:1999wi}
  J.~M.~Cline, G.~D.~Moore, G.~Servant,
  ``Was the electroweak phase transition preceded by a color broken phase?,''
  Phys.\ Rev.\  {\bf D60}, 105035 (1999).
  [hep-ph/9902220] (38 citations)

\bibitem{Cline:2000kb}
  J.~M.~Cline, K.~Kainulainen,
  ``A New source for electroweak baryogenesis in the MSSM,''
  Phys.\ Rev.\ Lett.\  {\bf 85}, 5519-5522 (2000).
  [hep-ph/0002272] (72 citations)

\bibitem{Cline:2000nw}
  J.~M.~Cline, M.~Joyce, K.~Kainulainen,
  ``Supersymmetric electroweak baryogenesis,''
  JHEP {\bf 0007}, 018 (2000).
  [hep-ph/0006119] (130 citations)


%\cite{Chen:2009dm}
\bibitem{Chen:2009dm}
  F.~Chen, J.~M.~Cline, A.~R.~Frey,
  ``A New twist on excited DM: Implications for INTEGRAL, PAMELA/ATIC/PPB-BETS, DAMA,''
  Phys.\ Rev.\  {\bf D79}, 063530 (2009).
  [arXiv:0901.4327 [hep-ph]]. (36 citations)

%\cite{Chen:2009ab}
\bibitem{Chen:2009ab}
  F.~Chen, J.~M.~Cline, A.~R.~Frey,
  ``Nonabelian DM: Models and constraints,''
  Phys.\ Rev.\  {\bf D80}, 083516 (2009).
  [arXiv:0907.4746 [hep-ph]]. (35 citations)

%\cite{Chen:2009av}
\bibitem{Chen:2009av}
  F.~Chen, J.~M.~Cline, A.~Fradette, A.~R.~Frey, C.~Rabideau,
  ``Exciting DM in the galactic center,''
  Phys.\ Rev.\  {\bf D81}, 043523 (2010).
  [arXiv:0911.2222 [hep-ph]]. (9 citations)

%\cite{Cline:2010ag}
\bibitem{Cline:2010ag}
  J.~M.~Cline, A.~C.~Vincent, W.~Xue,
  ``Leptons from DM Annihilation in Milky Way Subhalos,''
  Phys.\ Rev.\  {\bf D81}, 083512 (2010).
  [arXiv:1001.5399 [astro-ph.CO]]. (14 citations)

%\cite{Cirelli:2010nh}
\bibitem{Cirelli:2010nh}
  M.~Cirelli, J.~M.~Cline,
  ``Can multistate DM annihilation explain the high-energy cosmic ray lepton anomalies?,''
  Phys.\ Rev.\  {\bf D82}, 023503 (2010).
  [arXiv:1005.1779 [hep-ph]]. (17 citations)

\bibitem{Cline:2010kv}
  J.~M.~Cline, A.~R.~Frey, F.~Chen,
  ``Metastable DM mechanisms for INTEGRAL 511 keV $\gamma$ rays and DAMA/CoGeNT events,''
  Phys.\ Rev.\  {\bf D83}, 083511 (2011).
  [arXiv:1008.1784 [hep-ph]]. (11 citations)

\bibitem{Vincent:2010kv}
  A.~C.~Vincent, W.~Xue, J.~M.~Cline,
  ``Overcoming Gamma Ray Constraints with Annihilating DM in Milky Way Subhalos,''
  Phys.\ Rev.\  {\bf D82}, 123519 (2010).
  [arXiv:1009.5383 [hep-ph]]. (5 citations)

%\cite{Cline:2011zr}
\bibitem{Cline:2011zr}
  J.~M.~Cline, A.~R.~Frey,
  ``Minimal hidden sector models for CoGeNT/DAMA events,''
  Phys.\ Rev.\  {\bf D84} 075003 (2011) 
  [arXiv:1108.1391 [hep-ph]]. (3 citations)

%\cite{Cline:2011uu}
\bibitem{Cline:2011uu}
  J.~M.~Cline, A.~R.~Frey,
  ``Light DM versus astrophysical constraints,''
  submitted to Phys.\ Lett.\ {\bf B}
  [arXiv:1109.4639 [hep-ph]].

\bibitem{Cline:2006db}
  J.~M.~Cline, L.~Hoi,
  ``Inflationary potential reconstruction for a wmap running power spectrum,''
  JCAP {\bf 0606}, 007 (2006).
  [astro-ph/0603403]. (34 citations)

\bibitem{Hoi:2007sf}
  L.~Hoi, J.~M.~Cline,
  ``Testing for Features in the Primordial Power Spectrum,''
  Int.\ J.\ Mod.\ Phys.\  {\bf D18}, 1863-1888 (2009).
  [arXiv:0706.3887 [astro-ph]]. (10 citations)

%\cite{Cline:2003ve}
\bibitem{Cline:2003ve}
  J.~M.~Cline, P.~Crotty, J.~Lesgourgues,
  ``Does the small CMB quadrupole moment suggest new physics?,''
  JCAP {\bf 0309}, 010 (2003).
  [astro-ph/0304558]. (95 citations)

\bibitem{Burgess:2002ub}
  C.~P.~Burgess, J.~M.~Cline, F.~Lemieux, R.~Holman,
  ``Are inflationary predictions sensitive to very high-energy physics?,''
  JHEP {\bf 0302}, 048 (2003). 
  [arXiv:hep-th/0210233 [hep-th]].\\ (108 citations)

%\cite{BlancoPillado:2006he}
\bibitem{BlancoPillado:2006he}
  J.~J.~Blanco-Pillado, C.~P.~Burgess, J.~M.~Cline, C.~Escoda, M.~Gomez-Reino, R.~Kallosh, A.~D.~Linde, F.~Quevedo,
  ``Inflating in a better racetrack,''
  JHEP {\bf 0609}, 002 (2006).
  [hep-th/0603129]. (99 citations)

%\cite{BlancoPillado:2004ns}
\bibitem{BlancoPillado:2004ns}
  J.~J.~Blanco-Pillado, C.~P.~Burgess, J.~M.~Cline, C.~Escoda, M.~Gomez-Reino, R.~Kallosh, A.~D.~Linde, F.~Quevedo,
  ``Racetrack inflation,''
  JHEP {\bf 0411}, 063 (2004).
  [hep-th/0406230]. (175 citations)

\end{thebibliography}



\end{document}



Pat Scott is the most prominent ``star'' our group has had as a postdoc in
my recollection.  He functions already at the level of a junior
faculty member.  His research plan is extremely ambitious, but it is
also credible, as it is fully underway, involving numerous
collaborations with other researchers of international caliber.  Pat
has no need of supervision in the usual sense; rather he needs someone
who will facilitate his work as much as possible, by funding visits
from collaborators and providing manpower.  These I have been happy to
do during his first year with us.  Since I have a large number of
students it was not a hardship for me to spare my Ph.D.\ student
A.\ Vincent to work with him, and there is a possibility that M.Sc.\
student G.\ Dupuis will also join in one of his projects.   (He is
also working with W.\ Xue, a student of R.\ Brandenberger in our
group.) 
