\documentclass[10pt,twoside,twocolumn,a4paper]{article}

\usepackage{fancyhdr}
\usepackage[usenames,dvipsnames]{color}
\usepackage[colorlinks=true, linkcolor=BrickRed, citecolor=Blue, urlcolor=Blue, filecolor=Blue]{hyperref}
\usepackage{multibib}
\usepackage{ragged2e}
\usepackage{natbib_moderncvhacked_nospace}
\input{jdefs}

\newcites{papers}{Articles published or accepted in refereed journals}
\newcites{sub}{Articles submitted to refereed journals}
\newcites{proceedings}{Other refereed contributions (proceedings)}
\newcites{books}{Published Monographs}
\newcites{talks}{Invited Presentations}
\def\urltilda{\kern -.15em\lower .7ex\hbox{\~{}}\kern .04em}
\def\it{\slshape}
\def\tt{\ttfamily}
\def\bf{\bfseries}

\addtolength{\textheight}{0.33in}

\fancyhead[LE,RO]{\thepage}
\fancyhead[LO,RE]{\slshape Pat Scott}
\fancyhead[C]{Banting PDF CV Supporting Documents -- Research Contributions}
\fancyfoot[C]{}
\renewcommand{\headrulewidth}{0.4pt}
\renewcommand{\footrulewidth}{0pt}
\pagestyle{fancy}

\bibpunct{[}{]}{,}{n}{ }{,}

\author{Pat Scott}
\date{}

\makeatletter
\renewenvironment{thebibliography}[1]%
  {%
    \subsubsection*{\refname}%
    \small%
    \begin{list}{\bibliographyitemlabel}%
      {%
        \setlength{\topsep}{0pt}%
        %\setlength{\labelwidth}{\hintscolumnwidth}%
        %\setlength{\labelsep}{\separatorcolumnwidth}%
        \leftmargin\labelwidth%
        \advance\leftmargin\labelsep%
        \@openbib@code%
        \usecounter{enumiv}%
        \let\p@enumiv\@empty%
        \renewcommand\theenumiv{\@arabic\c@enumiv}}%
        \sloppy\clubpenalty4000\widowpenalty4000%
  }%
  {%
    \def\@noitemerr{\@latex@warning{Empty `thebibliography' environment}}%
    \end{list}%
  }
\makeatother

\begin{document}

\twocolumn[
{\bf Summary: 22 refereed publications since 2006, 736 citations, h-index: 11}\\
(Source: NASA Astrophysical Data System, Oct 14; see \href{http://www.physics.mcgill.ca/~patscott/publications}{www.physics.mcgill.ca/{\urltilda}patscott/publications})
\smallskip
\smallskip
]

\noindent\emph{Phys.\ Rev.\ Lett.: Physical Review Letters\\
Phys.\ Rev.\ D: Physical Review D (Particles, Fields, Gravitation and Cosmology)\\
JCAP: Journal of Cosmology and Astroparticle Physics\\
JHEP: Journal of High Energy Physics\\
ARA\&A:  Annual Review of Astronomy and Astrophysics\\
ApJ: The Astrophysical Journal (including Letters)\\
A\&A: Astronomy and Astrophysics\\
MNRAS: Monthly Notices of the Royal Astronomical Society\\
PoS: Proceedings of Science\\
Ap\&SS: Astrophysics and Space Science\\
Can.~J.~Phys.: Canadian Journal of Physics}

\subsection*{Part 1: List of Contributions}

\nocitepapers{Scott11,Ripken11,Akrami11coverage,Akrami11DD,Zackrisson10b,Zackrisson10a,Akrami09,Scott09c,AGSS,SS09,Scott09Ni,Scott09,Fairbairn08,ScottVII}
\bibliographystylepapers{JHEP_pat10}
\bibliographypapers{AbuGen,DMbiblio,SUSYbiblio}
\nocitesub{Bringmann11}
\bibliographystylesub{JHEP_pat10}
\bibliographysub{DMbiblio}
\nociteproceedings{ScottCRF,Rydberg10,GASS2,GASS3,GASS1,Scott09b,Scott08b,Scott08a}
\bibliographystyleproceedings{JHEP_pat10}
\bibliographyproceedings{AbuGen,DMbiblio,SUSYbiblio,CosmoSF}
\nocitebooks{PScottThesis}
\bibliographystylebooks{JHEP_pat10}
\bibliographybooks{DMbiblio}
%\setcounter{NAT@ctr}{0}
\nocitetalks{Fermilab11,TAUP11,TeVPA11,UTAustin,CRF2010,Nordic2010,Hamburg10,Lund10,Imperial10,Leicester10,Hamburg09,DarkStarsWorkshop,EAFAII,SEAS,ASABok}
\bibliographystyletalks{JHEP_pat10}
\bibliographytalks{talks}

\subsubsection*{+41 contributed presentations not shown (see \href{http://www.physics.mcgill.ca/~patscott/talks}{www.physics.mcgill.ca/{\urltilda}patscott/talks})}

\vspace{4mm}
\subsection*{Part 2: Research Contributions}

\cite{Scott09c} \textbf{Scott}, Conrad, Edsj{\"o}, Bergstr{\"o}m, Farnier \&  Akrami, {\it{Direct constraints on minimal supersymmetry from Fermi-LAT observations of the dwarf galaxy Segue 1}},  {\em \jcap} {\bf 1} (2010) 31.

\hfill 37 citations
\vspace{2mm}

In this paper, we investigated the implications for supersymmetry of searches for gamma-rays from dark matter (DM) annihilation in the ultra-faint dwarf galaxy Segue 1, using the \textit{Fermi} Large Area Telescope (LAT).  This paper was the first to include results from either direct or indirect searches for dark matter in a global SUSY fit.  It was also the first published dark matter result from the \textit{Fermi}-LAT collaboration itself.  

The work generated substantial interest from both the DM indirect detection and SUSY global-fit communities.  Whilst the constraints on the parameter space were not especially strong, the paper demonstrated how DM searches could be used in SUSY global fits, without compromising on any of the technical aspects of the astronomical gamma-ray analysis.  The community recongized (as evidenced in e.g. the referee's comments) that such analyses are indeed the way of the future for DM indirect detection.  The paper continues to be well read and referenced, generating 37 citations to date in the $\sim$2 years since it was posted to the arXiv.
\vspace{4mm}

\noindent \cite{AGSS} Asplund, Grevesse, Sauval \& \textbf{Scott} (AGSS09), {\it {The chemical composition of the Sun}},  {\em \araa} {\bf 47} (2009) 481--522.

\hfill 452 citations
\vspace{2mm}

This article updated and compiled the present state of knowledge on the abundances of the chemical elements in the Sun, from Hydrogen up to and including Uranium ($Z=92$).  The composition of the Sun is a fundamental reference quantity in astrophysics, against which all other astronomical objects are measured.  The article has found wide-ranging use in all parts of astronomy and astrophysics, from Galactic chemical evolution, planet searches and solar system physics to cosmology, extragalactic astrophysics and astroparticle physics.  In just two years, AGSS09 (as it is known in the community) has generated over 450 citations, becoming one of the most highly cited articles of 2009.

This paper, published as an invited review article in the prestigious journal \textit{Annual Reviews of Astronomy \& Astrophysics}, in fact consisted almost entirely of original work.  For no other article in my career where I was not first author have I ever done nearly as much work as for AGSS09.  We reanalyzed the abundances of every element with identifiable lines in the solar spectrum.  I was personally responsible for atomic data, line selection and abundance calculation for all elements from Na ($Z=11$) to Zn ($Z=30$).  This was the first time anyone had published a homogeneous and complete analysis of elemental abundances in the Sun, using the same model atmosphere, spectra, line- and data-selection policies and treatment of errors for every element.  We employed a new 3D hydrodynamic model of the solar atmosphere (the first time many elements had had their solar spectra analyzed in 3D), extremely discerning selections of atomic data and line lists, and in many cases, corrections for departures from local thermodynamic equilibrium in the atomic level populations.  In a sense, the article was a `pre-review' of a series of 7 papers we are currently finalizing for publication in \textit{Astronomy \& Astrophysics}, where all of the details of the results quoted in AGSS09 will be presented.
\vspace{4mm}

\noindent \cite{Scott09} \textbf{Scott}, Fairbairn \& Edsj{\"o, {\it {Dark stars at the Galactic Centre - the main sequence}},  {\em \mnras} {\bf 394} (2009) 82--104.

\hfill 33 citations
\vspace{2mm}

`Dark stars', stars whose structure and evolution are dictated more by dark matter annihilation in their cores than nuclear burning, have been a hot topic of research in recent years.  This paper dealt with prospects for finding dark stars close to the Galactic centre.  The paper was significant because it gave the full technical details and results of the most interesting investigations we discussed in earlier papers \cite{Fairbairn08,Scott08a}.  This work was the first treatment of the effects of dark matter on the evolution of main-sequence stars (of any metallicity), using a full stellar evolution simulation.  It was one of the initial 5-6 papers that really launched the recent run of $\sim$$50$ publications on dark stars.  It also lead directly to the public `dark stellar evolution' code \textsf{DarkStars} \cite{Scott09b}.  Aside from being interesting for the results it presented (showing that low-mass stars on elliptical orbits near the Galactic centre might be expected to exist as dark stars), the paper remains the most comprehensive and detailed exposition of the physics involved in simulating the structure and evolution of dark stars.  It has thus become something of a standard reference work in the field, and has generated 33 citations to date.

\end{document}
