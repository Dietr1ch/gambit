\documentclass[11pt,oneside,onecolumn,a4paper]{article}

\usepackage{graphicx}
\usepackage[footnotesize,bf]{caption}
\usepackage{fancyhdr}
\usepackage[numbers,sort&compress]{natbib}
\usepackage[usenames,dvipsnames]{color}
\usepackage{enumitem}
\usepackage[colorlinks=true, linkcolor=BrickRed, citecolor=Blue, urlcolor=Blue, filecolor=Blue]{hyperref}
\input{jdefs.tex}

\fancyhead[LE,RO]{\thepage}
\fancyhead[LO,RE]{Pat Scott}
\fancyhead[C]{STFC Ernest Rutherford Grant}
\fancyfoot[C]{}
\renewcommand{\headrulewidth}{0.4pt}
\renewcommand{\footrulewidth}{0pt}
\pagestyle{fancy}

\author{Pat Scott}
\date{}

\setlist{nolistsep}

\begin{document}

\thispagestyle{fancy}

\centerline{\LARGE \textbf{Pathways to Impact}}\bigskip 

Referring to the Impact Summary, there are three groups that stand to benefit from my proposed research program.  The first two can benefit via Knowledge Exchange, the third via Outreach activities.  Here I outline the strategies that I plan to employ in order to realise these benefits.

\section{Workers and firms in other quantitative fields}
The novel optimisation and data-mining software to be developed for and employed by the GAMBIT Project can in principle be useful in any field where optimisation problems must be solved.  Such applications include e.g. maximising returns from investments or efficiencies in engineering projects.  Any company that can solve its optimisation problems more quickly, or find better solutions than they otherwise would have with an inferior algorithm, stands to benefit financially from employing improved numerical optimisation methods.  These applications therefore provide potential commercialisation opportunities for the algorithms that will come out of the GAMBIT Project.  

I will investigate these opportunities with Imperial Innovations, the technology incubator office at Imperial College, under their Technology Transfer scheme.  Probably the most straightforward route to commercialisation will be to license the software to for-profit organisations interested in using it in their business.  Note that although I indicate elsewhere that the entire GAMBIT software suite will be publicly available, the commercialisable sections of it will be released under a licence that makes their use free of charge only in non-profit and academic settings.  Companies intending to use them to generate profit will be expected to purchase licenses.  

I will also investigate the prospect of deploying the software directly myself in industry, as part of a consultancy program facilitated by Imperial Consultants.  Ideally these two routes would be followed in parallel, i.e.\ licensing the software to a company and consulting for them myself on its best use.

\section{Policy makers and research councils}
The GAMBIT Project will enable quantitative comparisons of different proposed experiments, in terms of their expected reach into viable parameter spaces of theories beyond the Standard Model.  I will perform such calculations and provide them to the STFC Science Committee, to be used in evaluating different experimental proposals.  These comparisons will be directly useful in determining the most efficient future infrastructure spending in particle physics and astrophysics.  

Note that this is being done already in the USA's SNOWMASS decadal particle physics exercise, where the results of the Stanford Rizzo group's supersymmetric scans have been used extensively as a justification for recommending certain proposed experiments over others.  Those analyses suffer from severe statistical and numerical shortcomings (see Table 1 in my attached Case for Support).  The GAMBIT Project will provide far more rigorous, statistically justified recommendations than those on which many SNOWMASS recommendations have been based, giving the UK a substantial competitive advantage in terms of strategically funding searches for physics beyond the Standard Model into the future.

I will later also explore the possibility of joining the STFC Science Committee myself.  I will further investigate the prospects for making direct representations to policy makers (i.e.\ politicians and senior public servants) on the basis of the results obtained with GAMBIT, in order to advocate for certain experimental facilities over others.  The UK Government Science Advisory Panel would seem an appropriate conduit via which to attempt to do this.

\section{The general public}
I outlined a substantial outreach project based on global fits in my ERF application (see p17 in the attached 2012 application).  This involves the development of an interactive, touchscreen-based educational computer game for children, designed to illustrate the roles of different experiments around the world in the search for new particles.  The idea is to have first and second-year physics and computer science students work together on developing the game under the guidance of myself, the PDRA and the PhD student funded by this grant -- and then to take it on tour to various primary schools as part of the existing Imperial Astrophysics outreach efforts in schools.  This will increase dialogue and skill sharing between physics and computer science undergrads, increasing the professional prospects of both groups upon graduation.  It will also encourage an enthusiasm for physics amongst the school children, hopefully contributing to an increase in tertiary enrolments in mathematics and the sciences in the future.

In order to maximise the contribution of this grant to the outreach program of my ERF I will consider communications skills strongly in ranking studentship and postdoctoral candidates.  In this way I plan to select outgoing individuals who will contribute positively to outreach activities.

\end{document}
