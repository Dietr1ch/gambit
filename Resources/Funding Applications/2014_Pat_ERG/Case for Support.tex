\documentclass[11pt,oneside,twocolumn,a4paper]{article}

\usepackage{graphicx}
\usepackage[footnotesize,bf]{caption}
\usepackage{fancyhdr}
\usepackage[numbers,sort&compress]{natbib}
\usepackage[usenames,dvipsnames]{color}
\usepackage{enumitem}
\usepackage[colorlinks=true, linkcolor=BrickRed, citecolor=Blue, urlcolor=Blue, filecolor=Blue]{hyperref}
\input{jdefs.tex}

\definecolor{darkblue}{rgb}{0,0,0.3}

\addtolength{\topmargin}{-0.8in}
\addtolength{\textheight}{2.0in}
\addtolength{\textwidth}{0.95in}
\addtolength{\oddsidemargin}{-0.45in}
\setlength{\headheight}{26pt}
\setlength{\headsep}{0.15in}

\fancyhead[LE,RO]{p.\ \thepage}
\fancyhead[LO,RE]{Pat Scott}
\fancyhead[C]{\textcolor{darkblue}{Case for Support: Global searches for new Higgs physics with GAMBIT}}
\fancyfoot[C]{}
\renewcommand{\headrulewidth}{0.4pt}
\renewcommand{\footrulewidth}{0pt}
\pagestyle{fancy}

\author{Pat Scott}
\date{}

\setlist{nolistsep}

\begin{document}

This ERG will produce the most comprehensive analysis to date of theories for new particles related to the Higgs boson, including models where Higgs and dark matter are intimately connected.  These are the models most likely to be found or tested in the next round of LHC and dark matter experiments.

\subsubsection*{Context: The GAMBIT Project}

The Standard Model (SM) of particle physics is incomplete: it cannot explain dark matter (DM), neutrino masses, the excess of matter over antimatter, or apparent hierarchies between different energy scales.  Explanations for these must come from some physics Beyond the Standard Model (BSM).  Experiments provide many tests of new physics, including accelerator searches for new particles, studies of rare processes, and searches for scattering, annihilation and decay of DM.  The most powerful way to test theories is to perform a combined analysis of the data from \textit{all} experiments, so as to cover all avenues for discovering or ruling out new physics.  This must include all uncertainties, correlations and theoretical nuances self-consistently, and be compared to the predictions of many different theories.  My ERF addresses this challenge head-on by creating a novel framework for global statistical analysis of BSM theories, using all relevant data from astrophysics and particle physics to determine which theories best match reality.

The accuracy of my results will depend on treating experimental/observational searches, backgrounds and SM uncertainties in a detailed way.  To this end, over the last year and a half I have successfully formed a collaboration of 25 world-leading experimentalists and theorists, dedicated to developing the new global-fitting framework and including sophisticated experimental likelihoods in it: P. Athron, C.\ Balazs, T.\ Bringmann, A.\ Buckley, J.\ Conrad, J.\ M.\ Cornell, L.\ A.\ Dal, J.\ Edsj\"o, B.\ Farmer, P.\ Jackson, L.\ Hsu, A.\ Krislock, A.\ Kvellestad, N.\ Mahmoudi, G.\ Martinez, M.\ Pato, A.\ Putze, A.\ Raklev, C.\ Rogan, A.\ Saavedra, C.\ Savage, N.\ Serra, C.\ Weniger, M.\ White and myself.  This project is named GAMBIT, after the \textcolor{red}{G}lobal \textcolor{red}{a}nd \textcolor{red}{M}odular \textcolor{red}{B}SM \textcolor{red}{I}nference \textcolor{red}{T}ool that we are busy creating.  GAMBIT includes current members of ATLAS, LHCb, IceCube, CDMS, DM-ICE, DARWIN, Fermi-LAT, AMS-02, HESS, CTA and (soon) CMS, and boasts a formidable group of 15 theorists.

Development of the core GAMBIT framework is already well progressed.  Table \ref{comparison} gives a technical comparison of GAMBIT and other BSM scanning codes.

With the creation of GAMBIT, the scope and potential of my ERF have expanded dramatically.  I will hire a PDRA and a PhD student with this ER Grant, to seed a dedicated subgroup in GAMBIT focussed on the interface between DM and Higgs physics.  We will carry out the most comprehensive analyses to date of non-standard Higgs theories and ``Higgs-portal'' DM models (where DM interacts with the SM only via Higgs or related particles).

\begin{table*}
\centering
\caption{Features of BSM scanning codes.  \textcolor{red}{Red} extensions of GAMBIT are supported by this ERG. $\pm$$\epsilon \equiv$ small variants thereof.}\vspace{-2mm}
\label{comparison}
\scriptsize
\begin{tabular}{p{15mm}|p{67mm}|p{19mm}|p{19mm}|p{19mm}|p{19mm}}

\hline
Aspect & GAMBIT & MasterCode & SuperBayeS & Fittino & Rizzo et al. \\

\hline
Design & Modular, Adaptive & Monolithic & Monolithic & ($\sim$)Monolithic & Monolithic \\

Statistics & Frequentist, Bayesian & Frequentist & Freq., Bayes. & Frequentist & None \\

Scanners & Differential evolution, genetic algorithms, random forests, t-walk, t-nest, particle swarm, nested sampling, MCMC, gradient descent & Nested sampling, MCMC, grad.\ descent & Nested sampling, MCMC & MCMC & None\newline(random) \\

Theories & (p)MSSM-25, CMSSM$\pm$$\epsilon$, GMSB, AMSB, gaugino mediation, E6MSSM, NMSSM, BMSSM, PQMSSM, effective operators, iDM, XDM, ADM, UED, \textcolor{red}{Higgs-portal DM, extended Higgs sectors} & CMSSM$\pm$$\epsilon$ & (p)MSSM-15, CMSSM$\pm$$\epsilon$, mUED & CMSSM$\pm$$\epsilon$ & (p)MSSM-19 \\

Astro-\newline particle & Event-level: IceCube, Fermi, LUX, XENON, CDMS, DM-ICE. Basic: $\Omega_{\rm DM}$, AMS-02, COUPP, KIMS, CRESST, CoGeNT, SIMPLE, PAMELA, Planck, HESS. Predictions: CTA, DARWIN, GAPS  & Basic: $\Omega_{\rm DM}$, LUX, XENON & Basic: $\Omega_{\rm DM}$, Fermi, IceCube, XENON & Basic: $\Omega_{\rm DM}$, Fermi, HESS, XENON & Event-level: Fermi.\newline Basic: $\Omega_{\rm DM}$, IceCube, CTA \\

LHC & ATLAS+CMS multi-analysis with neural net and fast detector simulation.  \textcolor{red}{Higgs multi-channel with correlations and no SM assumptions}. Full flavour inc. complete $B\to X_sll$ and $B\to K^*ll$ angular set. & ATLAS resim, HiggsSignals, basic flavour. & ATLAS direct sim, Higgs mass only, basic flavour. & ATLAS resim, HiggsSignals, basic flavour. & ATLAS+CMS+\newline Tevatron direct sim, basic flavour. \\

SM, theory and related uncerts. & $m_t$, $m_b$, $\alpha_{\rm s}$, $\alpha_{\rm EM}$, DM halo, hadronic matrix elements, detector responses, \textcolor{red}{QCD and EW corrections (LHC \& DM signal \& BG)}, astro BGs, cosmic ray hadronisation, coalescence and propagation. & $m_t$, $m_Z$, $\alpha_{\rm EM}$, hadronic matrix elements & $m_t$, $m_b$, $\alpha_{\rm s}$, $\alpha_{\rm EM}$, DM halo, hadronic matrix elems. & $m_t$ & None \\

\hline
\end{tabular}

\end{table*}



\subsubsection*{Objectives}

\begin{enumerate}
\setlength{\itemsep}{2pt}
\item To constrain or discover BSM physics in the Higgs sector by combining accelerator measurements with dark matter searches
\item To determine which Higgs-portal dark matter and non-Standard Model Higgs sector theories are preferred by data
\item To determine if the observed Higgs boson shows any evidence of non-Standard Model behaviour
\end{enumerate}\smallskip


\subsubsection*{Timeliness}

\noindent The discovery of a Higgs boson provides an invaluable probe of BSM physics, as well as DM.  Comparing Higgs production and decay at the LHC, we can test BSM theories involving other particles that interact with the Higgs, by constraining the coupling of the Higgs to SM particles and ``invisibles'' like DM\textcolor{blue}{$^1$}.  Theories with such invisible particles typically involve extended ``Higgs sectors'' (the part of a theory that includes the Higgs and associated particles).

Higgs physics has exploded in recent years, with huge amounts of experimental data and proposed extensions to the SM based on non-standard Higgs sectors.  The global fit program in my ERF application only included the few most constraining Higgs channels, as is done in essentially all other current analyses.  However, with so much new, complementary data, such a wealth of new models to consider, and the GAMBIT program, there is now the opportunity to expand the scope of my research program to also encompass extremely detailed global fits to Higgs-sector phenomenology.  The recent discovery of a Higgs will mean that this is the sector in which our knowledge of particle physics is driven forwards most quickly in the near future.  Such an analysis is therefore urgently needed for directing experimental BSM searches, as well as essentially all GAMBIT BSM analyses.

In particular, detailed analysis of non-standard Higgs sectors will be an essential ingredient in a corresponding treatment of Higgs-portal DM models.  This class of DM models is now highly compelling because\begin{enumerate}\setlength{\itemsep}{2pt}
\item Discovery of a Higgs has opened the door for precision Higgs-sector measurements at the LHC.
\item Astroparticle anomalies compatible with Higgs-portal DM have appeared in cosmic rays, gamma rays and DM direct detection.
\item Limiting DM-SM interactions to a Higgs-like mediator is theoretically appealing, as it provides a natural reason for the darkness of DM but invokes relatively few new particles.\end{enumerate}  Despite strong interest in such theories, so far no exhaustive, quantitative comparison of the myriad different Higgs-portal type models has appeared.  The reason is simple: such an analysis requires a massive amount of work to include all relevant data in detail, and to apply it to all such models within this broad class.  For this a tool like GAMBIT is required.


\subsubsection*{Implementation}

I will hire a \textbf{PDRA} experienced in DM and Higgs phenomenology, to take responsibility in GAMBIT for the Higgs-portal activities shown in \textcolor{red}{red} in Table \ref{comparison}.  A PDRA is needed for this because it will be a significant responsibility within GAMBIT, and requires a breadth of experience in BSM phenomenology not rapidly acquirable by a student.  The PDRA will:

\textbf{RA1}.\ \textit{Implement theories for Higgs-portal DM models and non-standard Higgs sectors}: Type I--III 2-Higgs Doublet Models (2HDMs) and their variants, singlet and triplet extensions with scalar, vector or fermionic DM, effective Higgs couplings and so-called `Higgs friends' models.  Components of the global analyses of these models will necessarily cover the entire gamut of model-dependence from extreme (e.g.\ LHC production cross-sections) to mild (e.g.\ $\gamma$-ray yields) to model-independent (e.g.\ IceCube likelihood calculations).  GAMBIT features a hierarchical model database and a unique modular design that together ensure that the model-dependence of every sub-calculation is carefully tracked.  The work is therefore highly future-proof, as it will facilitate analysis of other BSM Higgs sector theories that have not yet even been proposed; routines written for one analysis are automatically re-applied to future models wherever permitted by theory, even if such models were not actually known at the time of the first analysis.

\textbf{RA2}.\ \textit{Implement state-of-the-art LHC Higgs likelihoods.}  These will improve on existing work because they will be explicitly computed for all theories in \textbf{RA1}, without assuming that the Higgs couplings resemble the SM prediction, and with treatment of correlations between the myriad channels probed by both ATLAS and CMS.  Myself, the PDRA and GAMBIT ATLAS member C.\ Rogan (Harvard) will engage closely with the LHC Higgs Combination Group (LHC-HCG), in order to best utilise (and further develop) their RooStats software framework for the joint channel likelihood.  Rogan, as an ATLAS and ex-CMS member, will be directly involved in ATLAS Higgs work and ATLAS+CMS combination efforts.

The optimal data for this analysis (individual channels with correlations) are only partially available at 8\,TeV.  There is however already \textit{much} more information available in published ATLAS and CMS Higgs papers and the LHC-HCG's joint channel likelihood than has been incorporated into existing global fits\textcolor{blue}{$^2$} (which do not cover the models we will analyse anyway) -- so even if the PDRA were only to use current data, our global fits would still be novel.  At 14\,TeV though, both ATLAS and CMS will release data correlated across different analysis channels on masses, cross-sections and systematics, making the analyses outlined here truly ground-breaking.  GAMBIT ATLAS and CMS members will also strongly advocate within the collaborations for increased public release of data, and provide insider insight into information that is public in principle but little-known in practice.

The \textbf{PhD student} will focus on progressively more advanced extensions to GAMBIT in support of the Higgs-portal program.  The student will:

\textbf{PhD1}.\ \textit{Help develop the GAMBIT generic particle spectrum class} with myself, P.\ Athron and B.\ Farmer.  This class provides a unified interface for accessing model spectra in GAMBIT, and is to be populated by interfaces to spectrum generator codes.  The student will create some of these interfaces (to SoftSUSY, SPheno, etc), as well as a simple RGE code for BSM Higgs sectors (generated with FlexibleSUSY).  This will pave the way for GAMBIT to include vacuum stability of the Higgs potential as a physicality requirement in global fits (including the MSSM), an important aspect that has so far been completely neglected\textcolor{blue}{$^{3}$}.  

\textbf{PhD2}.\ \textit{Carry out a full global physics analysis with one of the Higgs-portal DM models} from \textbf{RA1}, using the code from \textbf{PhD1} and the rest of GAMBIT.

\textbf{PhD3}.\ \textit{Interface GAMBIT with MadGraph5\_aMC@NLO and FeynRules}.  This will allow automatic calculation of tree-level BSM cross-sections for generic Lagrangians, and inclusion of NLO QCD corrections to both cosmic ray production processes and SM backgrounds in LHC searches.  This will complement electroweak corrections in GAMBIT's indirect dark matter routines (via DarkSUSY and PPPC4DMID), and extensions into NLO QCD-corrected relic densities with DM@NLO.  Including such loop corrections and SM uncertainties in GAMBIT will provide valuable feedback to the theory community about the relative importances of different corrections, as no other framework will be able to illustrate their global implications for real-world model constraints as completely as GAMBIT.

I will be developing the other GAMBIT software modules in parallel with the work of this ERG, so I will work closely with the PDRA and student to integrate their routines into the main codebase.  I will also provide theoretical guidance, having recently published highly-cited work on Higgs-portal DM\textcolor{blue}{$^4$} and many global fit papers in the MSSM\textcolor{blue}{$^5$} (which is itself a form of 2HDM).  Both the student and PDRA will also be able to draw on the extensive theoretical and experimental expertise of the rest the GAMBIT Collaboration.  Their projects will interact directly, as each will carry out global fits of closely-related models (\textbf{RA1} and \textbf{PhD2}).  This will place the PDRA in a perfect position to gain mentoring experience.  The student's work will benefit the PDRA, as the PDRA will be able to use the spectral code developed by the student in \textbf{PhD1} to implement specific Higgs-sector theories.  The work in this ERG will also broaden the PDRA and student's skillsets by way of their exposure to the full gamut of physics involved in GAMBIT.

The 20\% of my time spent on this project will enhance (not detract from) my ERF, as the work of all three of us will inform and optimise the scientific and coding strategies of the others.  Expanding the range of models, likelihoods and observables dealt with in global fits will also progress GAMBIT's goal of carrying out global analyses of as many theories as possible.

\medskip\noindent\textit{Timeline}
\begin{itemize}\setlength{\itemsep}{2pt}
\item Yr 1: Complete GAMBIT spectrum objects and interface to spectrum generators
\item Yr 1: Construct detailed likelihoods for all Higgs production and decay channels probed by the LHC, including correlations
\item Yr 2: Implement Higgs-portal DM and non-SM Higgs sector theories
\item Yr 2: Carry out Higgs-portal DM and brief 8\,TeV BSM Higgs global fits
\item Yr 2: Update to 14\,TeV LHC Higgs data
\item Yr 3: Update to 14\,TeV LHC non-Higgs data
\item Yr 3: Carry out full global Higgs-portal DM and 14\,TeV BSM Higgs sector analyses (including updated non-LHC data) 
\end{itemize}\vspace{1mm}

I have assumed startup of the 14\,TeV LHC in 2015; if it is delayed until e.g.\ mid 2016, the 8\,TeV global fits can be made more comprehensive.


\subsubsection*{Justification of costs}

The PDRA's salary is the Imperial College base rate.  Travel funds are for the PDRA, student and I to present the new work at conferences (to promote awareness of GAMBIT and our research efforts in the theoretical and experimental communities), and to visit other GAMBIT institutes (to work with other GAMBIT members).  We will host C.\ Rogan at Imperial for two weeks to work with him on the Higgs likelihoods.  The PDRA and student must attend the 9-monthly GAMBIT face-to-face meetings, so that they can explain their respective physics analyses and their additions to the Collaboration codebase, and solidify the international linkages they gain by being in GAMBIT.  I will also host one GAMBIT Collaboration Meeting in the UK.  This will highlight the UK's leading role in the Collaboration.  The PDRA and student will need laptops to develop and test global fit code, to give presentations at meetings, and to work on when travelling.  I will need a server to test and debug parallel global fit code, help host the GAMBIT repository, and manage backups.

\vspace{3.1mm}\noindent\textbf{Notes:} \textcolor{blue}{1}.\ PLB 723:340 2013 \textcolor{blue}{2}.\ e.g.\ $\sigma_{\rm ch1}$--$\sigma_{\rm ch2}$ and $m_H$--$\sigma$ likelihood maps; CMS-PAS-HIG-13-001,005 \textcolor{blue}{3}.\ arXiv:1407.2814 \textcolor{blue}{4}.\ PRD 88:055025 2013 \textcolor{blue}{5}.\ e.g.\ JCAP 11:057 2012, JHEP 04:057 2010, JCAP 01:031 2010.

\end{document}
